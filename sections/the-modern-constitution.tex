\section{The Modern Constitution}

\subsection{Overview}

\subsubsection{The New Deal and Economic Due Process}

\paragraph{Constitutional Adjudication in the Modern World 
(``Incorporation'')}

\begin{enumerate}
    \item \textbf{The Evolution of the Bill of Rights and Its 
    ``Incorporation'' against the States}
    \begin{enumerate}
        \item The Bill of Rights played a small role in the antebellum 
        era---e.g., the Court did not strike down the Alien and Sedition Act 
        of 1798.
        \item The Bill affirmed rights against the central government but not 
        against the states.
        \item The Civil War ``dramatized the need to limit abusive states'' 
        via the Fourteenth Amendment.\footnote{Casebook p. 487.}
        \item From the 1830s, antislavery activists developed the 
        \textbf{``declaratory''} interpretation of the Bill ``as affirming and 
        declaring pre-existing higher-law norms applicable to all government, 
        state as well as federal,'' rather than creating new or 
        federalism-based rules against the federal 
        government.\footnote{Casebook p. 487.}
        \item The Reconstruction Republicans who framed the Fourteenth 
        Amendment wanted to secure fundamental rights---privileges and 
        immunities---against violations by the states. The Bill of Rights was 
        a major source of these fundamental rights.
        \item ``By the end of the [19th] century almost all of the rights and 
        freedoms specified in the Founders' Bill had come to be applied 
        against state and local governments.''\footnote{Casebook p. 489.}
        \item Justice Cardozo: restrictions on Congress are tighter than 
        restrictions on the states---or, the states are not bound by the full 
        Bill of Rights.
        \item Justice Black: the theory of \textbf{``total incorporation''} 
        held that Fourteenth Amendment incorporated the full Bill and bound 
        the states to it.\footnote{Casebook p. 490.} 
        \item Justice Brennan: under \textbf{``selective incorporation''}, the 
        Court would decide which parts of the Bill were ``fundamental'' and 
        therefore binding on the states. The Warren court usually found 
        clauses of the Bill to be fundamental.
        \item ``Today, virtually all the Bill of Rights has come to apply with 
        equal vigor against state and local governments.''\footnote{Casebook 
        p. 490.}
    \end{enumerate}
\end{enumerate}

\paragraph{The Decline of Judicial Intervention Against Economic 
Regulation}

\begin{enumerate}
    \item After 1934, the government passed economic regulations to deal with 
    the depression. The Court struck down a half-dozen of them in 1935 and 
    1936. In 1937, Roosevelt's court packing threat led the Court to uphold 
    ``New Deal legislation against both economic due proces and 
    federalism-based challenges.''\footnote{Casebook p. 499.}
    \item \emph{Nebbia v. New York}: a storekeeper was convicted of selling 
    milk below the mandated minimum price. The Court held that regulations are 
    appropriate for industries ``affected with a public interest.'' ``The 
    phrase `affected with a public interest' can, in the nature of things, 
    mean no more than that an industry, for adequate reason, is subject to 
    control for the public good.''\footnote{Casebook p. 500.} However, the 
    Court did not make a clean break from the \emph{Lochner} doctrine that 
    courts should not interfere with private relations unless the business was 
    ``affected by a public interest.''
\end{enumerate}

\paragraph{1935--1937}

\begin{enumerate}
    \item \emph{Morehead v. New York ex rel. Tipaldo}: a minimum wage law for 
    women was invalid, on authority of \emph{Adkins v. Children's Hospital}. 
    ``~.~.~.~the State is without power by any form of legislation to 
    prohibit, change or nullify contracts between employers and adult women 
    workers as to the amount of wages to be paid.''\footnote{Casebook p. 511.}
    \item After Roosevelt's court packing scheme, the Court changed course and 
    overruled \emph{Adkins}.
\end{enumerate}

\paragraph{Limiting Judicial Review: \emph{United States v. Carolene 
Products}}

How can the Court justify judicial review after the 1937 changes in 
constitutional thought? In \emph{West Coast Hotel}, the Court held that 
``regulation which is reasonable in relation to its subject and is adopted in 
the interests of the community is due process.''\footnote{Casebook p. 516.} In 
\emph{Carolene Products}, the Court held that regulation is constitutional if 
it rest on ``some rational basis.'' The Court granted broad discretion to 
Congress on economic regulations and limited its own powers of judicial 
review. In ``footnote four,'' the Court reframed the basis for judicial review 
as (1) fidelity to the Constitution's actual text and (2) protecting 
democratic rights.

\begin{enumerate}
    \item The Filled Milk Act prohibited interstate commerce in milk 
    containing vegetable fat in place of milk fat. Carolene, a filled milk 
    manufacturer, challenged the constitutionality of the act under the 
    commerce clause and the Fifth Amendment.
    \item Justice Stone:
    \begin{enumerate}
        \item The Court dismissed the Fifth Amendment claim. Congress had the 
        power to regulate products that posed a harm to public health, and it 
        had no obligation to regulate \emph{all} evils.
        \item Legislation regulating harmful things is not constitutional by 
        default. The court should accept Congress's factual presumptions 
        as true. The basis for the constitutionality of a statute is whether 
        it is rational: ``regulatory legislation affecting ordinary commercial 
        transactions is not to be pronounced unconstitutional unless in the 
        light of the facts made known or generally assumed it is of such a 
        character as to preclude the assumption that it rests upon some 
        rational basis within the knowledge and experience of the 
        legislators.''\footnote{Casebook p. 515.}
    \end{enumerate}
    \item Justice Black, concurring:
    \begin{enumerate}
        \item Carolene should have had the chance to prove that its product 
        was not injurious to public health. 
    \end{enumerate}
    \item ``Interest group pluralism'': the idea of the ``public interest'' 
    has little substance of its own; rather, it reflects the interests of the 
    current majority. The ``pluralist model assumes that no groups 
    persistently exercise inappropriate or unfair degrees of political power 
    in a democracy.''\footnote{Casebook p. 517.}
    \item Why should the Court ever strike down any legislation? Footnote four 
    offers an answer. First, if the legislation specifically contradicts the 
    Constitution (e.g., by violating the Bill of Rights), the Courts can find 
    it unconstitutional. Second, the Court can act to protect democratic civil 
    rights and certain ``discrete and insular minorities.'' There is much 
    controversy about how to determine when the Court should exercise judicial 
    review under this model.\footnote{Casebook p. 515 (footnote four), p. 
    517--18.}
\end{enumerate}

\paragraph{\emph{Williamson v. Lee Optical}}

\begin{enumerate}
    \item An Oklahoma law prevented anyone not licensed as an optometrist or 
    opthalmologist to work on glasses.
    \item Justice Douglas:
    \begin{enumerate}
        \item First: the District Court held that \S\ 2 (unlicensed people 
        cannot work on glasses without a prescription from someone licensed) 
        was invalid under the due process clause. The Supreme Court held that 
        the Oklahoma legislature was within its power to regulate in the 
        public interest.
        \item Second: the District Court held that part of \S\ 3, exempting 
        makers of ready-made classes, violated equal protection. [Supreme 
        Court's holding?]
        \item the District Court held that part of \S\ 3, regulating the sale 
        of ``optical appliances,'' violated the due process clause. The 
        Supreme Court held that ``an eyeglass frame is not used in 
        isolation~.~.~.~;~it is used with lenses; and lenses, pertaining as 
        they do to the human eye, enter the field of health. The legislature 
        is therefore empowered to regulate the frames alone.\footnote{Casebook 
        p. 521.}
        \item Fourth: the District Court held that part of \S\ 4, preventing 
        retailers from subletting to eye doctors, violated due process. The 
        Supreme Court held that the regulation aimed to ``free the profession, 
        to as great an extent as possible, from the taint of 
        commercialism.''\footnote{Casebook p. 522.} It was therefore a valid 
        exercise of the legislature's power to protect public safety.
    \end{enumerate}
    % TODO notes 522-27
\end{enumerate}

\subsubsection{The Commerce Clause}

\paragraph{Relaxation of Judicial Constraints on Congressional Power}

\begin{enumerate}
    \item \emph{\textbf{NLRB v. Jones \& Laughlin}}: the National Labor 
    Relations Act prohibited employers ``from engaging in unfair labor 
    practices affecting commerce.'' Jones \& Laughlin was accused of violating 
    the act by interfering with employees' organizing and collective 
    bargaining rights. The Court held that Jones \& Laughlin's activities 
    would have an indirect but significant impact on commerce: ``Although 
    activities may be intrastate in character when separately considered, if 
    they have such a close and substantial relation to interstate commerce 
    that their control is essential or appropriate to protect that commerce 
    from burdens and obstructions, Congress cannot be denied the power to 
    exercise that control.''\footnote{Casebook p. 550.}
\end{enumerate}

\paragraph{\emph{United States v. Darby}}

\begin{enumerate}
    \item The Fair Labor Standards Act prescribed minimum wage and maximum 
    hour requirements for employees engaged in the production of goods related 
    to interstate commerce. Darby, a Georgia lumber manufacturer, was accused 
    of violating the act.
    \item The district court quashed the government's indictment. The 
    government appealed.
    \item Justice Stone divide his opinion into two parts.
    \item Part 1 (prohibition of shipment of proscribed goods in interstate 
    commerce under \S\ 15(a)(1)):
    \begin{enumerate}
        \item ``~.~.~.~the only question arising under the commerce clause 
        with respect to such shipments is whether Congress has the 
        constitutional power to prohibit them.''
        \item The government argued that Congress acted ``under the guise of 
        regulation of interstate commerce.'' But its real aim was to regulate 
        hours and wages.\footnote{Casebook p. 551.}
        \item Held: Congress's freedom to exercise its commerce clause powers 
        do not depend whether ``its exercise is attended by the same incidents 
        which attend the exercise of the police power of the 
        states.''\footnote{Casebook p. 551.} Or: Congress is free to exercise 
        police power via the commerce clause as long as the commerce clause 
        exercise is valid on its own.
        \item \emph{Hammer v. Dagenhart}, which held that Congress's commerce 
        clause powers were limited to regulation of things with ``some harmful 
        or deleterious property,'' is now overruled.
    \end{enumerate}
    \item Part 2 (validity of the wage and hour requirements under \S\S\ 
    15(a)(2), 6, and 7):
    \begin{enumerate}
        \item Darby's employees were not themselves engaged in interstate 
        commerce. The question was whether employees engaged in the production 
        of goods for interstate commerce were within the reach of Congress's 
        regulatory power.
        \item Congress intended the act not only to prevent transportation of 
        the prohibited product, but also to ``stop the initial step toward 
        transportation, production with the purpose of so transporting 
        it.''\footnote{Casebook p. 552.} Such regulation was within Congress's 
        power.
        \item Congress also appropriately exercised its commerce clause power 
        in coordinating economic activity between the states by limiting ``a 
        method or kind of competition in interstate commerce which it has in 
        effect condemned as \enquote{unfair}.''\footnote{Casebook p. 553.}
    \end{enumerate}
\end{enumerate}

\paragraph{\emph{Wickard v. Filburn}}

\begin{enumerate}
    \item The Agricultural Adjustment Act of 1938 mandated maximum allotments 
    of wheat for farmers. Filburn was accused of violating the act by growing 
    wheat for personal use in excess of the allotment.
    \item Justice Jackson:
    \begin{enumerate}
        \item ``~.~.~.~even if appellee's activity be local and though it may 
        not be regarded as commerce, it may still, whatever its nature, be 
        reached by Congress if it exerts a \textbf{substantial economic effect 
        on interstate commerce}~.~.~.~''\footnote{Casebook pp. 553--54.}
        \item The act in question was meant both to regulate the amount of 
        wheat on the market and ``the extent~.~.~.~to which one may forestall 
        resort to the market by producing to meet his own 
        needs.''\footnote{Casebook p. 554.} If everyone produced personal 
        wheat as Filburn did, it would have a substantial impact on the 
        interstate wheat market. Therefore, personal wheat production was 
        within Congress's regulatory power under the commerce clause.
    \end{enumerate}
\end{enumerate}

\paragraph{Post-\emph{Hammer} Issues}

\begin{enumerate}
    \item The \emph{Darby} court moved away from the \emph{Hammer} world in 
    which Congress could only regulate items it saw as 
    \enquote{noxious\enquote{in themselves}}.
    \item Does \emph{Darby} mean that Congress is free to regulate interstate 
    commerce for noneconomic reasons? Or does it mean that courts should 
    not try to discern Congress's motives as long as the regulation is a valid 
    exercise of its commerce clause power?\footnote{Casebook pp. 554--55.}
    \item \emph{Darby} and \emph{Wickard}, both unanimous, might suggest that 
    Congress enjoyed unlimited powers to regulate the national economy. But 
    both cases involved issues of economic competition where only Congress 
    could solve the coordination problem between states. States may have 
    wanted to implement similar economic regulation but couldn't because of 
    the prisoner's dilemma.
\end{enumerate}

\paragraph{On Constitutional Revolution: Ackerman vs. Balkin}

\begin{enumerate}
    \item Bruce Ackerman:
    \begin{enumerate}
        \item Theory of ``constitutional moments'': at significant points, 
        Americans ``amend'' the Constitution outside of Article V---e.g., 
        Reconstruction, New Deal. Political momentum transforms the Court, 
        which reinterprets constitutional principles.\footnote{Casebook pp. 
        556--57.}
        \item Constitutional moments are ``the epitome of democratic 
        self-governance.''\footnote{Casebook p. 557.}
        % TODO article V or Article V?
        \item Constitutional moments ``establish new standards for legitimacy 
        and correctness.''\footnote{Casebook p. 558.}
    \end{enumerate}
    \item Jack Balkin and Sanford Levinson:
        \item Theory of ``partisan entrenchment'': the executive packs the 
        judiciary with like-minded judges, producing significant changes in 
        constitutional interpretation over time.\footnote{Casebook p. 557.}
        \item Change can be gradual or dramatic. The process is ``roughly but 
        imperfectly democratic'' because in the long run, courts become 
        responsive to political majorities.\footnote{Casebook p. 558.}
        \item Change is not necessarily legitimate or correct.
\end{enumerate}


\subsection{The Modern Equal Protection Clause: Race}

\subsubsection{Racial Discrimination and National Security}

\paragraph{Ethnic Diversity and the United States: \emph{Chae Chan Ping v. 
United States}}

\begin{enumerate}
    \item Ping lived in San Francisco for 12 years. He returned to China, and 
    when he left the US, he obtained a certificate granting him reentry. While 
    he was away, Congress passed an amendment to the Chinese Exclusion Act, 
    which annulled Ping's certificate and his right to return. The district 
    court rejected his claim that the law restrained him of his liberty.
    \item Justice Field:
    \begin{enumerate}
        \item The act in question was based on ``a well-founded 
        apprehension~.~.~.~that a limitation to the immigration of certain 
        classes from China was essential to the peace of the community on the 
        Pacific coast, and possibly to the preservation of our civilization 
        there.''\footnote{Casebook p. 399.}
        \item Chinese immigration had been ``approaching the character of an 
        Oriental invasion, and was a menace to our 
        civilization.''\footnote{Casebook p. 400.}
        \item Ping argued that the act in question violated a labor treaty 
        with China.
        \item The Court here held that there was ``nothing in the treaties 
        between China and the United States to impair the validity of the act 
        of congress of October 1, 1988 [which prevented Ping's 
        return].''\footnote{Casebook p. 402.} Moreover, Congress would have 
        had the power to override an earlier treaty.
        \item Congress has the constitutional power to ``legitimately control 
        all individuals or governments within the American 
        territory.''\footnote{Casebook p.  403.} Field's opinion 
        quoted no specific constitutional authority for Congress's power to 
        regulate immigration. Instead, its authority derives ``from the nature 
        of sovereignty itself.''\footnote{Casebook p. 405.}
    \end{enumerate}
\end{enumerate}

\paragraph{\emph{Korematsu v. United States}}

\begin{enumerate}
    \item Facts:
    \begin{enumerate}
        \item December 7, 1941: Pearl Harbor.
        \item February 19, 1942---Executive Order 9066: From fear of espionage 
        and sabotage, all people of Japanese descent, whether American 
        citizens or not, were forced to relocate to internment camps.
        \item Later orders enforced 9066. Fred Korematsu was convicted of 
        violating the exclusion order.
    \end{enumerate}
    \item Justice Black:
    \begin{enumerate}
        \item Laws that restrict the rights of a single racial group are 
        ``immediately suspect'' but not unconstitutional by default.
        \item Here, military necessity validated an otherwise 
        unconstitutional act, as in \emph{Hirabashi}.\footnote{Casebook p. 
        968.}
        \item ``Korematsu was not excluded from the Military Area because 
        of hostility to him or his race. He \emph{was} excluded because 
        we are at war with the Japanese Empire~.~.~.~''\footnote{Casebook 
        p. 969.}
    \end{enumerate}
    \item Justice Frankfurter, concurring:
    \begin{enumerate}
        \item ``~.~.~.~the validity of action under the war power must be 
        judged wholly in the context of war.''\footnote{Casebook p. 969.}
    \end{enumerate}
    \item Justice Roberts, dissenting:
    \begin{enumerate}
        \item Confining someone to a ``concentration camp'' becuase of his 
        ancestry violates his constitutional rights.\footnote{Casebook p. 
        970.}
    \end{enumerate}
    \item Justice Murphy, dissenting:
    \begin{enumerate}
        \item ``The judicial test of whether the Government, on plea of 
        military necessity, can validly deprive an individual of any of 
        his constitutional rights is \textbf{whether the deprivation is 
        reasonable}.''\footnote{Casebook p. 970.} The order here ``clearly 
        does not meet that test.''
        \item The justification for the order was based ``mainly upon 
        questionable racial and sociological grounds not ordinarily within 
        the realm of expert military judgment~.~.~.~''\footnote{Casebook 
        p. 971.}
        \item ``I dissent, therefore, from this legalization of 
        racism.''\footnote{Casebook p. 972.}
    \end{enumerate}
    \item Justice Jackson, dissenting:
    \begin{enumerate}
        \item If the government had enacted such a law during peacetime, 
        the Court would refuse to enforce it. By the majority's logic, any 
        law passed during war under the guise of military necessity would 
        be constitutional by default.
        \item ``I should hold that a civil court cannot be made to enforce 
        an order which violates constitutional limitations even if it is a 
        reasonable exercise of military authority. The courts can exercise 
        only the judicial power, can only apply law, and must abide by the 
        Constitution, or they cease to be civil courts and instead become 
        instruments of military policy.''\footnote{Casebook p. 974.}
    \end{enumerate}
    % TODO: notes 975--981
\end{enumerate}

% \subsubsection{\emph{Brown}}
% 
% \paragraph{Background}
% 
% \begin{enumerate}
%     \item % TODO 893-98
% \end{enumerate}
% 
% \paragraph{\emph{Brown v. Board of Education}}
% 
% \begin{enumerate}
%     \item % TODO 898-902
% \end{enumerate}
% 
% \paragraph{\emph{Parents Involved in Community Schools v. Seattle School 
% District No. 1}}
% 
% \begin{enumerate}
%     \item % TODO excerpt
% \end{enumerate}
% 
% \paragraph{A ``Dissent'' from \emph{Brown}}
% 
% \begin{enumerate}
%     \item % TODO 902-04
% \end{enumerate}
% 
% \paragraph{Originalism and Anti-Discrimination Law}
% 
% \begin{enumerate}
%     \item % TODO 912-15
% \end{enumerate}
% 
% \paragraph{Beyond Originalism?}
% 
% \begin{enumerate}
%     \item % TODO 920-23
% \end{enumerate}
% 
% \subsubsection{\emph{Brown II} and \emph{Hernandez}}
% 
% \paragraph{Reflections on the Opinion in \emph{Brown}}
% 
% \begin{enumerate}
%     \item % TODO 923-24
% \end{enumerate}
% 
% \paragraph{The Enduring Significance of \emph{Brown}}
% 
% \begin{enumerate}
%     \item % TODO 925-27
% \end{enumerate}
% 
% \paragraph{Four Decades of School Desegregation (\emph{Brown II}, 
% \emph{Green}, \emph{Swann})}
% 
% \begin{enumerate}
%     \item % TODO 928-36
% \end{enumerate}
% 
% \paragraph{The Turning Point---Interdistrict Relief (\emph{Milliken v. 
% Bradley})}
% 
% \begin{enumerate}
%     \item % TODO 941-43
% \end{enumerate}
% 
% \paragraph{An Era of Retrenchment}
% 
% \begin{enumerate}
%     \item % TODO 943-45
% \end{enumerate}
% 
% \subsubsection{Strict Scrutiny (Anticlassification vs. Antisubordination)}
% 
% \paragraph{\emph{Hernandez v. Texas}}
% 
% \begin{enumerate}
%     \item % TODO 1010-14
% \end{enumerate}
% 
% \paragraph{The Antidiscrimination Principle}
% 
% \begin{enumerate}
%     \item % TODO 956-59
% \end{enumerate}
% 
% \paragraph{\emph{Loving v. Virginia}}
% 
% \begin{enumerate}
%     \item % TODO 959-66
% \end{enumerate}
% 
% \paragraph{What Justifies the Suspect Classification Standard?}
% 
% \begin{enumerate}
%     \item % TODO 984-90
% \end{enumerate}
% 
% \subsubsection{The Intent Standard, Version 1}
% 
\paragraph{What is a Race-Dependent Decision?}

\begin{enumerate}
    \item Race-dependent decisions are not always overt. Brest et al. use 
    ``race-motivated'' to refer to a decision in which race motivates the 
    decisionmaker in any way.\footnote{Casebook p. 1021.}
    \item What obligations does the antidiscrimination principle impose on 
    initial decisionmakers?
    \item When should courts inquire into whether a decision was 
    race-dependent?
    \item \textbf{\emph{Yick Wo v. Hopkins}---Discriminatory Administration of 
    an Otherwise ``Neutral'' Statute}: the San Francisco Board of Supervisors 
    granted laundromat permits to nearly all of 80 Caucasion applicants and 
    none of 200 Chinese applicants. The Court held that there was no reason 
    for it except to express hostility to a group.
    \item \textbf{``Queue Ordinance Case'': \emph{Ho Ah Kow v. Nunan}---The 
    Race-Dependent Decision to Adopt a Nonracially Specific Regulation or 
    Law}: every male prisoner's hair had to be cut to within one inch. The 
    practical effect was to coerce Chinese people into paying fines, because 
    they dreaded the cutting of the queue (a braid they held sacred).
    \begin{enumerate}
        \item \textbf{Gomillion v. Lightfoot}): the Alabama legislature 
        changed the boundaries of Tuskegee from a square to ``an uncouth 
        twenty-eight-sided figure.''\footnote{Casebook p. 1023.} The sole 
        purpose was to segregate voters by race.
    \end{enumerate}
    \item \textbf{\emph{Gaston County v. United States}---Transferred De Jure 
    Discrimination}: a voting literacy test disproportionately disenfranchised 
    blacks. The Court held that years of inferior education meant that blacks 
    were less equipped to pass the test, so the test was discriminatory.
\end{enumerate}
 
\paragraph{Title VII and Disparate Impact: \emph{Griggs v. Duke Power}}

Title VII prohibits employers from requiring job applicants to have high 
school diplomas and pass a general intelligence test without showing that 
those criteria predict job performance.

\begin{enumerate}
    \item An employer required job applicants to have a high school diploma 
    and pass a general intelligence test.
    \item Justice Burger: the Civil Rights Act ``proscribes not only overt 
    discrimination but also practices that are fair in form, but 
    discriminatory in operation. The touchstone is business necessity. If an 
    employment practice which operates to exclude Negroes cannot be shown to 
    be related to job performance, the practice is prohibited.
\end{enumerate}

\paragraph{\emph{Washington v. Davis}}

The Court declined to read the \emph{Griggs} ``disparate impact'' standard 
into the Fourteenth Amendment.

\begin{enumerate}
    \item The Civil Service Commission issued a personnel test to applicants 
    who sought to become police officers. Plaintiffs sued to invalidate the 
    test on the grounds that it violated the Fifth Amendment.
    \item The appellate court held for the plaintiffs, incorporating the 
    \emph{Griggs} interpretation of Title VII into the Fifth and (by 
    implication) the Fourteenth Amendments.
    \item Justice White:
    \begin{enumerate}
        \item Something is not unconstitutional ``\emph{solely} because it has 
        a racially disproportionate impact.''\footnote{Casebook p. 1027.}
        \item De jure segregation is not the same as de facto segregation.
        \item ``Disproportionate impact is not irrelevant, but it is not the 
        sole touchstone of an invidious racial discrimination forbidden by the 
        Constitution.''\footnote{Casebook p. 1028.}
    \end{enumerate}
    \item Justice Stevens, concurring: there is not such a bright line between 
    discriminatory purpose and discriminatory impact. ``For normally the actor 
    is presumed to have intended the natural consequences of his 
    deeds.''\footnote{Casebook p. 1030.}
\end{enumerate}

\subsubsection{\emph{Griggs} as a Constitutional Principle and \emph{Griggs} 
versus \emph{Davis}}

\begin{enumerate}
    \item \emph{Davis} exempted policies from violating the Fifth and 
    Fourteenth amendments on disparate impact alone. As a solution, what about 
    a requirement that policymakers confront the racial impact of their 
    policies?\footnote{Casebook p. 1034.} Environmental impact statements are 
    analogous and are already required.
    \item There is little legislative history to support the ``disparate 
    treatment'' interpretation of Title VII from \emph{Griggs}. So why the 
    different outcome from \emph{Davis}? One possible explanation is that 
    Title VII applies only to employment, while the Fourteenth Amendment 
    applies to all scenarios. Another possible explanation is that the police 
    department in \emph{Davis} was full of black officers, including the 
    chief.
\end{enumerate}

\paragraph{The \emph{Arlington Heights} Factors}

\begin{enumerate}
    \item Factors that courts can use to determine when government decisions 
    are racially motivated:
    \begin{enumerate}
        \item The impact of the action, including patterns that emerge.
        \item The decision's historical background.
        \item The sequence events leading up to the decision.
        \item ``Departures from the normal procedural 
        sequence.''\footnote{Casebook p. 1040.}
        \item Substantive departures from normal procedure.
        \item Legislative or administrative history.
    \end{enumerate}
\end{enumerate}
 
% \subsubsection{Colorblindness}
% 
% \paragraph{\emph{United Jewish Organizations (UJO)}}
% 
% \begin{enumerate}
%     \item % TODO handout
% \end{enumerate}
% 
% \paragraph{\emph{University of California v. Bakke, Part I}}
% 
% \begin{enumerate}
%     \item % TODO handout
% \end{enumerate}
% 
% \paragraph{\emph{Richmond v. Croson}}
% 
% \begin{enumerate}
%     \item % TODO 1081-1109
% \end{enumerate}
% 
% \paragraph{\emph{Adarand v. Pena}}
% 
% \begin{enumerate}
%     \item % TODO 1109-13 (skim)
% \end{enumerate}
% 
% \subsubsection{The Intent Standard, Version 2: \emph{Feeney} and After}
% 
% \paragraph{Discussion Following \emph{Washington v. Davis}}
% 
% \begin{enumerate}
%     \item % TODO 1031-33
% \end{enumerate}
% 
% \paragraph{Commentaries on the Intent Standard}
% 
% \begin{enumerate}
%     \item % TODO 1035-39
% \end{enumerate}
% 
% \paragraph{\emph{McCleskey v. Kemp}}
% 
% \begin{enumerate}
%     \item % TODO 1055-63
% \end{enumerate}
% 
% \paragraph{Memo from Justice Scalia on \emph{McCleskey} Draft Opinion}
% 
% \begin{enumerate}
%     \item % TODO handout
% \end{enumerate}
% 
% \subsubsection{Affirmative Action in Higher Education (Diversity)}
% 
% \paragraph{\emph{University of California v. Bakke, Part II}}
% 
% \begin{enumerate}
%     \item % TODO handout
% \end{enumerate}
% 
% \paragraph{\emph{Grutter v. Bolinger}}
% 
% \begin{enumerate}
%     \item % TODO 1120-42
% \end{enumerate}
% 
% \paragraph{\emph{Gratz v. Bollinger}}
% 
% \begin{enumerate}
%     \item % TODO 1142-51 (skim)
% \end{enumerate}
% 
% \subsubsection{Race and Public Policy}
% 
% \paragraph{\emph{Parents Involved in Community Schools v. Seattle School 
% District No. 1}}
% 
% \begin{enumerate}
%     \item % TODO handout
% \end{enumerate}
% 
% \paragraph{\emph{Ricci v. DeStefano}}
% 
% \begin{enumerate}
%     \item % TODO handout from Brest 2011 supplement
% \end{enumerate}
% 
% \subsection{The Modern Equal Protection Clause: Gender}
% 
% \subsubsection{Intermediate Scrutiny}
% 
% \paragraph{Social Movements}
% 
% \begin{enumerate}
%     \item % TODO 1179-87
% \end{enumerate}
% 
% \paragraph{\emph{Frontiero v. Richardson}}
% 
% \begin{enumerate}
%     \item % TODO 1188-95
% \end{enumerate}
% 
% \paragraph{The Equal Rights Amendment}
% 
% \begin{enumerate}
%     \item % TODO 1195-1202
% \end{enumerate}
% 
% \subsubsection{Relevant Differences or Stereotypes}
% 
% \paragraph{What Justifies Special Constitutional Scrutiny}
% 
% \begin{enumerate}
%     \item % TODO 1202-13
% \end{enumerate}
% 
% \paragraph{What Does Intermediate Scrutiny Prohibit? \emph{Craig v. Boren}}
% 
% \begin{enumerate}
%     \item % TODO 1213-19
% \end{enumerate}
% 
% \paragraph{On Sex, Gender, and Sexual Orientation}
% 
% \begin{enumerate}
%     \item % TODO 1224-26
% \end{enumerate}
% 
% \paragraph{Jury Service: \emph{J.E.B. v. Alabama}}
% 
% \begin{enumerate}
%     \item % TODO 1226-28
% \end{enumerate}
% 
% \subsubsection{Not Sex-Based Differences}
% 
% \paragraph{\emph{Personnel Administrator of Massachusetts v. Feeney}}
% 
% \begin{enumerate}
%     \item % TODO 1262-71
% \end{enumerate}
% 
% \paragraph{Domestic Violence and Marital Rape}
% 
% \begin{enumerate}
%     \item % TODO 1271-76
% \end{enumerate}
% 
% \paragraph{\emph{Geduldig v. Aviello}}
% 
% \begin{enumerate}
%     \item % TODO + notes, 1276-81
% \end{enumerate}
% 
% \subsubsection{Permissible Sex-Based Differences}
% 
% \paragraph{\emph{Michael M. v. Superior Court of Sonoma}}
% 
% \begin{enumerate}
%     \item % TODO 1282-95
% \end{enumerate}
% 
% \subsubsection{Separate Facilities}
% 
% \paragraph{The VMI Case: \emph{United States v. Virginia}}
% 
% \begin{enumerate}
%     \item % TODO 1229-55
% \end{enumerate}
% 
% \subsubsection{Affirmative Action, Intersectionality, and Marriage}
% 
% \paragraph{Affirmative Action}
% 
% \begin{enumerate}
%     \item % TODO 1323-27
% \end{enumerate}
% 
% \paragraph{Discrimination against Women of Color}
% 
% \begin{enumerate}
%     \item % TODO 1258-59
% \end{enumerate}
% 
% \paragraph{Intermediate Scrutiny and Same-Sex Marriage}
% 
% \begin{enumerate}
%     \item % TODO 1219-24
% \end{enumerate}
% 
% \subsection{Modern Substantive Due Process}
% 
% \subsubsection{Implied Fundamental Rights: Contraception}
% 
% \paragraph{The Ninth Amendment}
% 
% \begin{enumerate}
%     \item % TODO 151-53
% \end{enumerate}
% 
% \paragraph{Antecedents of Fundamental Rights Adjudication}
% 
% \begin{enumerate}
%     \item % TODO 1339-42
% \end{enumerate}
% 
% \paragraph{\emph{Griswold v. Connecticut}}
% 
% \begin{enumerate}
%     \item % TODO 1342-55
% \end{enumerate}
% 
% \paragraph{Theories of Fundamental Rights Adjudication}
% 
% \begin{enumerate}
%     \item % TODO 1355-65
% \end{enumerate}
% 
% \subsubsection{Implied Fundamental Rights: Abortion}
% 
% \paragraph{\emph{Roe v. Wade}}
% 
% \begin{enumerate}
%     \item % TODO + discussion and note, 1279-81
% \end{enumerate}
% 
% \paragraph{Abortion and the Equal Protection Clause}
% 
% \begin{enumerate}
%     \item % TODO 1409-19
% \end{enumerate}
% 
% \subsubsection{Decisions After \emph{Roe}}
% 
% % TODO: generally: 1419-24
% 
% \paragraph{\emph{Planned Parenthood v. Case}}
% 
% \begin{enumerate}
%     \item % TODO 1224-57
% \end{enumerate}
% 
% \paragraph{\emph{Gonzales v. Carhart}}
% 
% \begin{enumerate}
%     \item % TODO from supplement
% \end{enumerate}
% 
% \subsubsection{Sexual Orientation and Due Process}
% 
% \paragraph{Sexuality and Sexual Orientation}
% 
% \begin{enumerate}
%     \item % TODO 1465-66
% \end{enumerate}
% 
% \paragraph{\emph{Bowers v. Hardwick}}
% 
% \begin{enumerate}
%     \item % TODO 1466-82
% \end{enumerate}
% 
% \subsubsection{Sexual Orientation and Equal Protection}
% 
% \paragraph{\emph{Romer v. Evans}}
% 
% \begin{enumerate}
%     \item % TODO 1505-1515
% \end{enumerate}
% 
% \subsubsection{Sexual Orientation and Due Process, Take 2}
% 
% \paragraph{\emph{Lawrence v. Texas}}
% 
% \begin{enumerate}
%     \item % TODO 1482-1505
% \end{enumerate}
% 
% \paragraph{Sexual Orientation as a Suspect Classification}
% 
% \begin{enumerate}
%     \item % TODO 1518-32
% \end{enumerate}
% 
% \subsubsection{Same-Sex Marriage}
% 
% \paragraph{\emph{California Marriage Cases}}
% 
% \begin{enumerate}
%     \item % TODO handout
% \end{enumerate}
% 
% \subsection{Other Suspect Classifications and Fundamental Rights}
% 
% \subsubsection{Wealth and Education (Substantive Equal Protection)}
% 
% \paragraph{\emph{San Antonio v. Rodriguez}}
% 
% \begin{enumerate}
%     \item % TODO 1623-41
% \end{enumerate}
% 
% \subsubsection{Alienage}
% 
% \paragraph{Citizenship and Alienage under the Equal Protection Clause}
% 
% \begin{enumerate}
%     \item % TODO 1156-60
% \end{enumerate}
% 
% \paragraph{\emph{Graham v. Richardson}}
% 
% \begin{enumerate}
%     \item % TODO 1160-63
% \end{enumerate}
% 
% \paragraph{\emph{Bernal v. Fainter}}
% 
% \begin{enumerate}
%     \item % TODO 1163-72
% \end{enumerate}
% 
% \paragraph{Regulation of Resident Aliens}
% 
% \begin{enumerate}
%     \item % TODO 1172-77
% \end{enumerate}
% 
% \paragraph{\emph{Plyler v. Doe}}
% 
% \begin{enumerate}
%     \item % TODO + note, 1641-47
% \end{enumerate}
% 
% \subsubsection{\emph{Arizona v. United States}}
% 
% \begin{enumerate}
%     \item % TODO handout
% \end{enumerate}
