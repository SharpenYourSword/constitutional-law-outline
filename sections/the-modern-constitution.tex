\section{The Modern Constitution}

\subsection{Overview}

\subsubsection{The New Deal and Economic Due Process}

\paragraph{Constitutional Adjudication in the Modern World 
(``Incorporation'')}

\begin{enumerate}
    \item \textbf{The Evolution of the Bill of Rights and Its 
    ``Incorporation'' against the States}
    \begin{enumerate}
        \item The Bill of Rights played a small role in the antebellum 
        era---e.g., the Court did not strike down the Alien and Sedition Act 
        of 1798.
        \item The Bill affirmed rights against the central government but not 
        against the states.
        \item The Civil War ``dramatized the need to limit abusive states'' 
        via the Fourteenth Amendment.\footnote{Casebook p. 487.}
        \item From the 1830s, antislavery activists developed the 
        \textbf{``declaratory''} interpretation of the Bill ``as affirming and 
        declaring pre-existing higher-law norms applicable to all government, 
        state as well as federal,'' rather than creating new or 
        federalism-based rules against the federal 
        government.\footnote{Casebook p. 487.}
        \item The Reconstruction Republicans who framed the Fourteenth 
        Amendment wanted to secure fundamental rights---privileges and 
        immunities---against violations by the states. The Bill of Rights was 
        a major source of these fundamental rights.
        \item ``By the end of the [19th] century almost all of the rights and 
        freedoms specified in the Founders' Bill had come to be applied 
        against state and local governments.''\footnote{Casebook p. 489.}
        \item Justice Cardozo: restrictions on Congress are tighter than 
        restrictions on the states---or, the states are not bound by the full 
        Bill of Rights.
        \item Justice Black: the theory of \textbf{``total incorporation''} 
        held that Fourteenth Amendment incorporated the full Bill and bound 
        the states to it.\footnote{Casebook p. 490.} 
        \item Justice Brennan: under \textbf{``selective incorporation''}, the 
        Court would decide which parts of the Bill were ``fundamental'' and 
        therefore binding on the states. The Warren court usually found 
        clauses of the Bill to be fundamental.
        \item ``Today, virtually all the Bill of Rights has come to apply with 
        equal vigor against state and local governments.''\footnote{Casebook 
        p. 490.}
    \end{enumerate}
\end{enumerate}

\paragraph{The Decline of Judicial Intervention Against Economic 
Regulation}

\begin{enumerate}
    \item After 1934, the government passed economic regulations to deal with 
    the depression. The Court struck down a half-dozen of them in 1935 and 
    1936. In 1937, Roosevelt's court packing threat led the Court to uphold 
    ``New Deal legislation against both economic due proces and 
    federalism-based challenges.''\footnote{Casebook p. 499.}
    \item \emph{Nebbia v. New York}: a storekeeper was convicted of selling 
    milk below the mandated minimum price. The Court held that regulations are 
    appropriate for industries ``affected with a public interest.'' ``The 
    phrase `affected with a public interest' can, in the nature of things, 
    mean no more than that an industry, for adequate reason, is subject to 
    control for the public good.''\footnote{Casebook p. 500.} However, the 
    Court did not make a clean break from the \emph{Lochner} doctrine that 
    courts should not interfere with private relations unless the business was 
    ``affected by a public interest.''
\end{enumerate}

\paragraph{1935--1937}

\begin{enumerate}
    \item \emph{Morehead v. New York ex rel. Tipaldo}: a minimum wage law for 
    women was invalid, on authority of \emph{Adkins v. Children's Hospital}. 
    ``~.~.~.~the State is without power by any form of legislation to 
    prohibit, change or nullify contracts between employers and adult women 
    workers as to the amount of wages to be paid.''\footnote{Casebook p. 511.}
    \item After Roosevelt's court packing scheme, the Court changed course and 
    overruled \emph{Adkins}.
\end{enumerate}

\paragraph{\emph{United States v. Carolene Products}}

\begin{enumerate}
    \item The Filled Milk Act prohibited interstate commerce in milk 
    containing vegetable fat in place of milk fat. Carolene, a filled milk 
    manufacturer, challenged the constitutionality of the act under the 
    commerce clause and the Fifth Amendment.
    \item Justice Stone:
    \begin{enumerate}
        \item The Court dismissed the Fifth Amendment claim. Congress had the 
        power to regulate products that posed a harm to public health, and it 
        had no obligation to regulate \emph{all} evils.
        \item Legislation regulating harmful things is not constitutional by 
        default.
        \item The court should accept Congress's factual presumptions 
        as true, and instead focus on whether the legislation is rational. In 
        this case it was.
    \end{enumerate}
    \item Justice Black, concurring:
    \begin{enumerate}
        \item Carolene should have had the chance to prove that its product 
        was not injurious to public health. 
    \end{enumerate}
    % TODO: notes pp. 516--520
\end{enumerate}

\paragraph{\emph{Williamson v. Lee Optical}}

\begin{enumerate}
    \item An Oklahoma law prevented anyone not licensed as an optometrist or 
    opthalmologist to work on glasses.
    \item Justice Douglas:
    \begin{enumerate}
        \item First: the District Court held that \S\ 2 (unlicensed people 
        cannot work on glasses without a prescription from someone licensed) 
        was invalid under the due process clause. The Supreme Court held that 
        the Oklahoma legislature was within its power to regulate in the 
        public interest.
        \item Second: the District Court held that part of \S\ 3, exempting 
        makers of ready-made classes, violated equal protection. [Supreme 
        Court's holding?]
        \item the District Court held that part of \S\ 3, regulating the sale 
        of ``optical appliances,'' violated the due process clause. The 
        Supreme Court held that ``an eyeglass frame is not used in 
        isolation~.~.~.~;~it is used with lenses; and lenses, pertaining as 
        they do to the human eye, enter the field of health. The legislature 
        is therefore empowered to regulate the frames alone.\footnote{Casebook 
        p. 521.}
        \item Fourth: the District Court held that part of \S\ 4, preventing 
        retailers from subletting to eye doctors, violated due process. The 
        Supreme Court held that the regulation aimed to ``free the profession, 
        to as great an extent as possible, from the taint of 
        commercialism.''\footnote{Casebook p. 522.} It was therefore a valid 
        exercise of the legislature's power to protect public safety.
    \end{enumerate}
    % TODO notes 522-27
\end{enumerate}

% \subsubsection{The Commerce Clause}
% 
% \paragraph{Relaxation of Judicial Constraints on Congressional Power}
% 
% \begin{enumerate}
%     \item % TODO 549-51
% \end{enumerate}
% 
% \paragraph{\emph{United States v. Darby}}
% 
% \begin{enumerate}
%     \item % TODO 551-58
% \end{enumerate}
% 
% \subsection{The Modern Equal Protection Clause: Race}
% 
% \subsubsection{Racial Discrimination and National Security}
% 
% \paragraph{Ethnic Diversity and the United States: \emph{Chae Chan Ping v. 
% United States}}
% 
% \begin{enumerate}
%     \item % TODO 398-405
% \end{enumerate}
% 
% \paragraph{\emph{Korematsu v. United States}}
% 
% \begin{enumerate}
%     \item % TODO 967-81 and note
% \end{enumerate}
% 
% \subsubsection{\emph{Brown}}
% 
% \paragraph{Background}
% 
% \begin{enumerate}
%     \item % TODO 893-98
% \end{enumerate}
% 
% \paragraph{\emph{Brown v. Board of Education}}
% 
% \begin{enumerate}
%     \item % TODO 898-902
% \end{enumerate}
% 
% \paragraph{\emph{Parents Involved in Community Schools v. Seattle School 
% District No. 1}}
% 
% \begin{enumerate}
%     \item % TODO excerpt
% \end{enumerate}
% 
% \paragraph{A ``Dissent'' from \emph{Brown}}
% 
% \begin{enumerate}
%     \item % TODO 902-04
% \end{enumerate}
% 
% \paragraph{Originalism and Anti-Discrimination Law}
% 
% \begin{enumerate}
%     \item % TODO 912-15
% \end{enumerate}
% 
% \paragraph{Beyond Originalism?}
% 
% \begin{enumerate}
%     \item % TODO 920-23
% \end{enumerate}
% 
% \subsubsection{\emph{Brown II} and \emph{Hernandez}}
% 
% \paragraph{Reflections on the Opinion in \emph{Brown}}
% 
% \begin{enumerate}
%     \item % TODO 923-24
% \end{enumerate}
% 
% \paragraph{The Enduring Significance of \emph{Brown}}
% 
% \begin{enumerate}
%     \item % TODO 925-27
% \end{enumerate}
% 
% \paragraph{Four Decades of School Desegregation (\emph{Brown II}, 
% \emph{Green}, \emph{Swann})}
% 
% \begin{enumerate}
%     \item % TODO 928-36
% \end{enumerate}
% 
% \paragraph{The Turning Point---Interdistrict Relief (\emph{Milliken v. 
% Bradley})}
% 
% \begin{enumerate}
%     \item % TODO 941-43
% \end{enumerate}
% 
% \paragraph{An Era of Retrenchment}
% 
% \begin{enumerate}
%     \item % TODO 943-45
% \end{enumerate}
% 
% \subsubsection{Strict Scrutiny (Anticlassification vs. Antisubordination)}
% 
% \paragraph{\emph{Hernandez v. Texas}}
% 
% \begin{enumerate}
%     \item % TODO 1010-14
% \end{enumerate}
% 
% \paragraph{The Antidiscrimination Principle}
% 
% \begin{enumerate}
%     \item % TODO 956-59
% \end{enumerate}
% 
% \paragraph{\emph{Loving v. Virginia}}
% 
% \begin{enumerate}
%     \item % TODO 959-66
% \end{enumerate}
% 
% \paragraph{What Justifies the Suspect Classification Standard?}
% 
% \begin{enumerate}
%     \item % TODO 984-90
% \end{enumerate}
% 
% \subsubsection{The Intent Standard, Version 1}
% 
% \paragraph{What is a Race-Dependent Decision? (\emph{Yick Wo}, \emph{Queue 
% Ordinance Case}, \emph{Gomillion}, \emph{Gaston County})}
% 
% \begin{enumerate}
%     \item % TODO 1020-24
% \end{enumerate}
% 
% \paragraph{\emph{Griggs v. Duke Power}}
% 
% \begin{enumerate}
%     \item % TODO 1024-26
% \end{enumerate}
% 
% \paragraph{\emph{Washington v. Davis}}
% 
% \begin{enumerate}
%     \item % TODO 1026-31
% \end{enumerate}
% 
% \subsubsection{\emph{Griggs} as a Constitutional Principle and \emph{Griggs} 
% versus \emph{Davis}}
% 
% \begin{enumerate}
%     \item % TODO 1033-34
% \end{enumerate}
% 
% \paragraph{The \emph{Arlington Heights} Factors}
% 
% \begin{enumerate}
%     \item % TODO 1039-40
% \end{enumerate}
% 
% \subsubsection{Colorblindness}
% 
% \paragraph{\emph{United Jewish Organizations (UJO)}}
% 
% \begin{enumerate}
%     \item % TODO handout
% \end{enumerate}
% 
% \paragraph{\emph{University of California v. Bakke, Part I}}
% 
% \begin{enumerate}
%     \item % TODO handout
% \end{enumerate}
% 
% \paragraph{\emph{Richmond v. Croson}}
% 
% \begin{enumerate}
%     \item % TODO 1081-1109
% \end{enumerate}
% 
% \paragraph{\emph{Adarand v. Pena}}
% 
% \begin{enumerate}
%     \item % TODO 1109-13 (skim)
% \end{enumerate}
% 
% \subsubsection{The Intent Standard, Version 2: \emph{Feeney} and After}
% 
% \paragraph{Discussion Following \emph{Washington v. Davis}}
% 
% \begin{enumerate}
%     \item % TODO 1031-33
% \end{enumerate}
% 
% \paragraph{Commentaries on the Intent Standard}
% 
% \begin{enumerate}
%     \item % TODO 1035-39
% \end{enumerate}
% 
% \paragraph{\emph{McCleskey v. Kemp}}
% 
% \begin{enumerate}
%     \item % TODO 1055-63
% \end{enumerate}
% 
% \paragraph{Memo from Justice Scalia on \emph{McCleskey} Draft Opinion}
% 
% \begin{enumerate}
%     \item % TODO handout
% \end{enumerate}
% 
% \subsubsection{Affirmative Action in Higher Education (Diversity)}
% 
% \paragraph{\emph{University of California v. Bakke, Part II}}
% 
% \begin{enumerate}
%     \item % TODO handout
% \end{enumerate}
% 
% \paragraph{\emph{Grutter v. Bolinger}}
% 
% \begin{enumerate}
%     \item % TODO 1120-42
% \end{enumerate}
% 
% \paragraph{\emph{Gratz v. Bollinger}}
% 
% \begin{enumerate}
%     \item % TODO 1142-51 (skim)
% \end{enumerate}
% 
% \subsubsection{Race and Public Policy}
% 
% \paragraph{\emph{Parents Involved in Community Schools v. Seattle School 
% District No. 1}}
% 
% \begin{enumerate}
%     \item % TODO handout
% \end{enumerate}
% 
% \paragraph{\emph{Ricci v. DeStefano}}
% 
% \begin{enumerate}
%     \item % TODO handout from Brest 2011 supplement
% \end{enumerate}
% 
% \subsection{The Modern Equal Protection Clause: Gender}
% 
% \subsubsection{Intermediate Scrutiny}
% 
% \paragraph{Social Movements}
% 
% \begin{enumerate}
%     \item % TODO 1179-87
% \end{enumerate}
% 
% \paragraph{\emph{Frontiero v. Richardson}}
% 
% \begin{enumerate}
%     \item % TODO 1188-95
% \end{enumerate}
% 
% \paragraph{The Equal Rights Amendment}
% 
% \begin{enumerate}
%     \item % TODO 1195-1202
% \end{enumerate}
% 
% \subsubsection{Relevant Differences or Stereotypes}
% 
% \paragraph{What Justifies Special Constitutional Scrutiny}
% 
% \begin{enumerate}
%     \item % TODO 1202-13
% \end{enumerate}
% 
% \paragraph{What Does Intermediate Scrutiny Prohibit? \emph{Craig v. Boren}}
% 
% \begin{enumerate}
%     \item % TODO 1213-19
% \end{enumerate}
% 
% \paragraph{On Sex, Gender, and Sexual Orientation}
% 
% \begin{enumerate}
%     \item % TODO 1224-26
% \end{enumerate}
% 
% \paragraph{Jury Service: \emph{J.E.B. v. Alabama}}
% 
% \begin{enumerate}
%     \item % TODO 1226-28
% \end{enumerate}
% 
% \subsubsection{Not Sex-Based Differences}
% 
% \paragraph{\emph{Personnel Administrator of Massachusetts v. Feeney}}
% 
% \begin{enumerate}
%     \item % TODO 1262-71
% \end{enumerate}
% 
% \paragraph{Domestic Violence and Marital Rape}
% 
% \begin{enumerate}
%     \item % TODO 1271-76
% \end{enumerate}
% 
% \paragraph{\emph{Geduldig v. Aviello}}
% 
% \begin{enumerate}
%     \item % TODO + notes, 1276-81
% \end{enumerate}
% 
% \subsubsection{Permissible Sex-Based Differences}
% 
% \paragraph{\emph{Michael M. v. Superior Court of Sonoma}}
% 
% \begin{enumerate}
%     \item % TODO 1282-95
% \end{enumerate}
% 
% \subsubsection{Separate Facilities}
% 
% \paragraph{The VMI Case: \emph{United States v. Virginia}}
% 
% \begin{enumerate}
%     \item % TODO 1229-55
% \end{enumerate}
% 
% \subsubsection{Affirmative Action, Intersectionality, and Marriage}
% 
% \paragraph{Affirmative Action}
% 
% \begin{enumerate}
%     \item % TODO 1323-27
% \end{enumerate}
% 
% \paragraph{Discrimination against Women of Color}
% 
% \begin{enumerate}
%     \item % TODO 1258-59
% \end{enumerate}
% 
% \paragraph{Intermediate Scrutiny and Same-Sex Marriage}
% 
% \begin{enumerate}
%     \item % TODO 1219-24
% \end{enumerate}
% 
% \subsection{Modern Substantive Due Process}
% 
% \subsubsection{Implied Fundamental Rights: Contraception}
% 
% \paragraph{The Ninth Amendment}
% 
% \begin{enumerate}
%     \item % TODO 151-53
% \end{enumerate}
% 
% \paragraph{Antecedents of Fundamental Rights Adjudication}
% 
% \begin{enumerate}
%     \item % TODO 1339-42
% \end{enumerate}
% 
% \paragraph{\emph{Griswold v. Connecticut}}
% 
% \begin{enumerate}
%     \item % TODO 1342-55
% \end{enumerate}
% 
% \paragraph{Theories of Fundamental Rights Adjudication}
% 
% \begin{enumerate}
%     \item % TODO 1355-65
% \end{enumerate}
% 
% \subsubsection{Implied Fundamental Rights: Abortion}
% 
% \paragraph{\emph{Roe v. Wade}}
% 
% \begin{enumerate}
%     \item % TODO + discussion and note, 1279-81
% \end{enumerate}
% 
% \paragraph{Abortion and the Equal Protection Clause}
% 
% \begin{enumerate}
%     \item % TODO 1409-19
% \end{enumerate}
% 
% \subsubsection{Decisions After \emph{Roe}}
% 
% % TODO: generally: 1419-24
% 
% \paragraph{\emph{Planned Parenthood v. Case}}
% 
% \begin{enumerate}
%     \item % TODO 1224-57
% \end{enumerate}
% 
% \paragraph{\emph{Gonzales v. Carhart}}
% 
% \begin{enumerate}
%     \item % TODO from supplement
% \end{enumerate}
% 
% \subsubsection{Sexual Orientation and Due Process}
% 
% \paragraph{Sexuality and Sexual Orientation}
% 
% \begin{enumerate}
%     \item % TODO 1465-66
% \end{enumerate}
% 
% \paragraph{\emph{Bowers v. Hardwick}}
% 
% \begin{enumerate}
%     \item % TODO 1466-82
% \end{enumerate}
% 
% \subsubsection{Sexual Orientation and Equal Protection}
% 
% \paragraph{\emph{Romer v. Evans}}
% 
% \begin{enumerate}
%     \item % TODO 1505-1515
% \end{enumerate}
% 
% \subsubsection{Sexual Orientation and Due Process, Take 2}
% 
% \paragraph{\emph{Lawrence v. Texas}}
% 
% \begin{enumerate}
%     \item % TODO 1482-1505
% \end{enumerate}
% 
% \paragraph{Sexual Orientation as a Suspect Classification}
% 
% \begin{enumerate}
%     \item % TODO 1518-32
% \end{enumerate}
% 
% \subsubsection{Same-Sex Marriage}
% 
% \paragraph{\emph{California Marriage Cases}}
% 
% \begin{enumerate}
%     \item % TODO handout
% \end{enumerate}
% 
% \subsection{Other Suspect Classifications and Fundamental Rights}
% 
% \subsubsection{Wealth and Education (Substantive Equal Protection)}
% 
% \paragraph{\emph{San Antonio v. Rodriguez}}
% 
% \begin{enumerate}
%     \item % TODO 1623-41
% \end{enumerate}
% 
% \subsubsection{Alienage}
% 
% \paragraph{Citizenship and Alienage under the Equal Protection Clause}
% 
% \begin{enumerate}
%     \item % TODO 1156-60
% \end{enumerate}
% 
% \paragraph{\emph{Graham v. Richardson}}
% 
% \begin{enumerate}
%     \item % TODO 1160-63
% \end{enumerate}
% 
% \paragraph{\emph{Bernal v. Fainter}}
% 
% \begin{enumerate}
%     \item % TODO 1163-72
% \end{enumerate}
% 
% \paragraph{Regulation of Resident Aliens}
% 
% \begin{enumerate}
%     \item % TODO 1172-77
% \end{enumerate}
% 
% \paragraph{\emph{Plyler v. Doe}}
% 
% \begin{enumerate}
%     \item % TODO + note, 1641-47
% \end{enumerate}
% 
% \subsubsection{\emph{Arizona v. United States}}
% 
% \begin{enumerate}
%     \item % TODO handout
% \end{enumerate}
