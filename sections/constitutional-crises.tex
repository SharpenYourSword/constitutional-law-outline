\section{Constitutional Crises}

\subsection{Reconstruction}

\subsubsection{Reconstruction Amendments}

\subsubsection{History of the Adoption of the Fourteenth Amendment}

\begin{enumerate}
    \item ``Black Codes'' discriminated against ostensibly free slaves, e.g., 
    prohibitions against weapons or liquor, or specific enforcement of labor 
    contracts.
    \item \textbf{Civil Rights Act of 1866}:
    \begin{enumerate}
        \item The \textbf{``civil rights formula''} prohibited discrimination 
        in ``civil rights or immunities.''\footnote{Casebook p. 303.}
        \item Specific rights were enumerated, but there was strong dispute 
        about the scope. Voting was widely understood to be excluded, but the 
        opposition worried that the scope would be construed broadly to end 
        (supposedly legitimate) discrimination practices like segregation.
        \item In addition worries about scope, the opposition also doubted 
        whether the Thirteenth Amendment---which only banned 
        slavery---authorized Congress to enact the civil rights bill.
        \item The civil rights formula was eventually struck, and Congress 
        enacted a bill that only protected a narrower set of enumerated rights 
        (contracts, evidence, property, etc.).\footnote{Casebook p. 307 n. 9.}
    \end{enumerate}
    \item \textbf{Fourteenth Amendment}:
    \begin{enumerate}
        \item John Bingham introduced the proposed amendment language in the 
        house.
        \item Democrats and conservative-leaning Republicans worried that the 
        amendment delegated too much power to Congress. Radical Republicans 
        worried that the amendment was not ``self-executing,'' i.e., its 
        actual implementation would depend on the whims of Congress.
        \item Many argued the amendment was the civil rights act in new 
        clothes. The scope of the ``privileges and immunities'' clause was 
        contested along the same lines at the civil rights formula in the 
        earlier act. Many of these concerns were justified, as proponents 
        made comments that the amendment's language \emph{should} be construed 
        broadly to end racial discrimination in all forms.\emph{Casebook pp. 
        308--09.}
        \item 
    \end{enumerate}
    \item Congress rejected other proposed amendments that explicitly required 
    color blindness. Does this mean that Congress did \emph{not} intend the 
    Equal Protection Clause to require color blindness in all situations?
\end{enumerate}

\subsubsection{The Fourteenth Amendment Limited}

\begin{enumerate}
    \item The Reconstruction amendments mainly aimed to end slavery, but they 
    also established protections for a broader range of rights. A key question 
    was, what rights did the new amendments guarantee?
    \item The right to vote was understood to be outside the scope of civil 
    rights, but others were strongly contested.
    \item The \emph{Slaughter-House Cases} tested the meaning of ``free labor'' 
    as a civil right.
\end{enumerate}

\paragraph{Scope of Civil Rights Protections: \emph{The Slaughter-House Cases}}

Is free labor a fundamental right? Do the Reconstruction amendments---the 
Fourteenth in particular---protect the right to labor, or does the power 
remain with the states?

\begin{enumerate}
    \item 1869: Louisiana wanted to move unpleasant slaughterhouses out of the 
    New Orleans city limits. It granted exclusive rights to the slaughtering 
    industry to a new corporation and required all New Orleans butchers to use 
    its facilities. 
    \item Justice Miller:
    \begin{enumerate}
        \item The butchers were not deprived of labor because they were free 
        to use the newly incorporated slaughterhouse.
        \item The state's police power authorized it to protect ``the general 
        interests of the community.''\footnote{Casebook p. 320.} Regulations of 
        the meat industry ``are among the most necessary and frequent exercise 
        of this power.''\footnote{Casebook p. 321.}
        \item The state has the power to incorporate a municipality, so it 
        should also have the power to form another corporation. The court did 
        not think that the monopoly privileges granted to this corporation 
        were ``especially odious or objectionable.''\footnote{Casebook p. 321.}
        \item \textbf{Argument 1}: the statute creates involuntary servitude.
        \begin{enumerate}
            \item The Reconstruction amendments were meant to end slavery. 
            The amendments have nothing to do with labor and property rights.
        \end{enumerate}
        \item \textbf{Argument 2}: the statute abridges privileges and 
        immunities of United States citizens.
        \begin{enumerate}
            \item The Fourteenth Amendment draws a clear distinction between 
            United States citizenship and state citizenship.
            \item The privileges and immunities clause\footnote{``No State 
            shall make or enforce any law which shall abridge the privileges 
            or immunities of citizens of the United States~.~.~.~'' U.S. 
            Const. amend. XIV, \S\ 1.} protects only \emph{federal} rights.
            only \emph{federal} rights.
            \item \emph{Corfield v. Coryell}: ``privileges and immunities'' 
            covers only a narrow range of ``fundamental'' rights.
            \item States are responsible for a large domain of civil rights. 
            The Fourteenth Amendment did not expand the power of Congress or 
            the Court to act as a ``perpetual censor upon all legislation of 
            the States, on the civil rights of their own 
            citizens~.~.~.~''\footnote{Casebook p. 324.} Such an 
            interpretation would ``fetter and degrade'' the states and 
            ``radically change[] the whole theory of the relations of the State 
            and Federal governments to each other~.~.~.~''\footnote{Casebook 
            p. 325.}
            \item Federal civil rights include asserting claims against the 
            government, right to assembly, right to petition for redress of 
            grievances, and habeus corpus.
        \end{enumerate}
        \item \textbf{Argument 3}: the statute deprives butchers of property 
        without due process.
        \begin{enumerate}
            \item This was not a deprivation of property. The due process 
            argument is irrelevant.\footnote{Casebook p. 325--26.}
        \end{enumerate}
        \item \textbf{Argument 4}: the statute denies equal protection.
        \begin{enumerate}
            \item No. The equal protection clause applies only to 
            slavery.\footnote{Casebook p. 326.}
        \end{enumerate}
    \end{enumerate}
    \item Justice Field, dissenting:
    \begin{enumerate}
        \item Police powers do not allow a state to violate constitutional 
        rights.
        \item Granting monopoly privileges to the new slaughterhouse 
        corporation did not promote the health of the city.
        \item Granting monopoly privileges to the government (e.g., ferries, 
        bridges) is different from granting monopoly privileges to private 
        entity in ``one of the ordinary trades or callings of 
        life.''\footnote{Casebook p. 326.} By the majority's rationale, 
        ``there is no monopoly, in the most odious form, which may not be 
        upheld.''\footnote{Casebook p. 327.}
        \item The Fourteenth Amendment \emph{does} protect U.S. citizens 
        against violations by state legislatures of ``common 
        rights'' (i.e., inalienable natural rights, like the right to property 
        and to pursue happiness)\footnote{Casebook p. 327.}
    \end{enumerate}
    \item Justice Bradley, dissenting:
    \begin{enumerate}
        \item Private property is a fundamental right. The freedom to choose a 
        profession is essential to that right.
        \item The original constitution prevented the federal government from 
        abridging fundamental rights. Now, the Fourteenth Amendment extended 
        the same protections to abridgement by the states.
        \item Ending slavery was not the sole purpose of the Reconstruction 
        amendments. ``It is futile to argue that none but persons of the 
        African race are intended to be benefited by this amendment. They may 
        have been the primary cause of the amendment, but its language is 
        general, embracing all citizens, and I think it was purposely so 
        expressed.''\footnote{Casebook p. 329.}
    \end{enumerate}
    \item Justice Swayne, dissenting:
    \begin{enumerate}
        \item The first eleven amendments put ``checks and limitations'' on 
        the federal government. The Reconstruction amendments limit state 
        power.
        \item ``Labor is property, and as such merits 
        protection~.~.~.~''\footnote{Casebook p. 330.}
        \item The Fourteenth Amendment granted the federal government power to 
        protect citizens' fundamental rights against oppression by the states. 
        This power is ``eminently conservative~.~.~.~bulwark of 
        defense~.~.~.~can never be made an engine of oppression~.~.~.~cannot 
        be abused.''\footnote{Casebook p. 330.}
        \item States are sometimes (often?) more oppressive than the federal 
        government---contrary to the founders' fears.
    \end{enumerate}
\end{enumerate}

% \subsubsection{Early Application of the Fourteenth Amendment to Women} 
% 
% \paragraph{Women's Citizenship in the Antebellum Period}
% 
% \begin{enumerate}
%     \item % TODO 164-68
% \end{enumerate}
% 
% \paragraph{\emph{Bradwell v. Illinois}}
% 
% \begin{enumerate}
%     \item % TODO 337-39
% \end{enumerate}
% 
% \paragraph{The ``New Departure'' and Women's Place in the Constitutional 
% Order}
% 
% \begin{enumerate}
%     \item % TODO 340-43
% \end{enumerate}
% 
% \paragraph{\emph{Minor v. Happersett}}
% 
% \begin{enumerate}
%     \item % TODO 343-46
% \end{enumerate}
% 
% \subsubsection{The Private Sphere and State Action} 
% 
% \paragraph{Establishment of the ``Separate but Equal'' Doctrine}
% 
% \begin{enumerate}
%     \item % TODO 357-58
% \end{enumerate}
% 
% \paragraph{\emph{The Civil Rights Cases}}
% 
% \begin{enumerate}
%     \item % TODO 373-85
% \end{enumerate}
% 
% \subsubsection{``Separate but Equal''}
% 
% \paragraph{\emph{Plessy v. Ferguson}}
% 
% \begin{enumerate}
%     \item % TODO 359-69
% \end{enumerate}
% 
% \paragraph{The Spirit of \emph{Plessy}}
% 
% \begin{enumerate}
%     \item % TODO 370-73
% \end{enumerate}
% 
% \subsection{Economic Rights and Structural Concerns}
% 
% \subsubsection{The Lochner Era: Substantive Due Process} 
% 
% \paragraph{Pressures for Intervention and the Rise of Substantive Due 
% Process, 1874--1890}
% 
% \begin{enumerate}
%     \item % TODO 412-15
% \end{enumerate}
% 
% \paragraph{\emph{Lochner v. New York}}
% 
% \begin{enumerate}
%     \item % TODO 417-31
% \end{enumerate}
% 
% \subsubsection{The Commerce Clause} 
% 
% \paragraph{Congressional Regulation of Interstate Commerce}
% 
% \begin{enumerate}
%     \item % TODO 435-37
% \end{enumerate}
% 
% \paragraph{\emph{Champion v. Ames}}
% 
% \begin{enumerate}
%     \item % TODO 437-41
% \end{enumerate}
% 
% \paragraph{\emph{Hammer v. Dagenhart}}
% 
% \begin{enumerate}
%     \item % TODO 441-45
% \end{enumerate}
% 
% \paragraph{Prisonner's Dilemmas}
% 
% \begin{enumerate}
%     \item % TODO 445-47
% \end{enumerate}
