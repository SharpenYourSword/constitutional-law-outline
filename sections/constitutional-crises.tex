\section{Constitutional Crises}

\subsection{Reconstruction}

\subsubsection{\emph{Dred Scott v. Sandford}}

\begin{enumerate}
    \item Can Congress ban slavery in the states? No.
    \item The Court held the Republican party's main platform 
    unconstitutional.
\end{enumerate}

\subsubsection{Reconstruction Amendments}

\subsubsection{History of the Adoption of the Fourteenth Amendment}

\begin{enumerate}
    \item ``Black Codes'' discriminated against ostensibly free slaves, e.g., 
    prohibitions against weapons or liquor, or specific enforcement of labor 
    contracts. They essentially outlawed normal employment for blacks.
    \item Convict lease system: maybe as destructive as slavery, if not 
    more so.
    \item \textbf{Civil Rights Act of 1866: what rights were protected?}
    \begin{enumerate}
        \item The \textbf{``civil rights formula''} prohibited discrimination 
        in ``civil rights or immunities.''\footnote{Casebook p. 303.}
        \item Specific rights were enumerated, but there was still strong 
        dispute about the scope of the civil rights formula. Voting was widely 
        understood to be excluded, but the opposition worried that the scope 
        would be construed broadly to end (supposedly legitimate) 
        discrimination practices like segregation.
        \item In addition worries about scope, the opposition also doubted 
        whether the Thirteenth Amendment---which only banned 
        slavery---authorized Congress to enact the civil rights bill.
        \item The civil rights formula was eventually struck, and Congress 
        enacted a bill that only protected a narrower set of enumerated rights 
        (contracts, evidence, property, etc.).\footnote{Casebook p. 307 n. 9.}
    \end{enumerate}
    \item \textbf{Fourteenth Amendment}:
    \begin{enumerate}
        \item John Bingham introduced the proposed amendment language in the 
        House.
        \item Democrats and conservative-leaning Republicans worried that the 
        amendment delegated too much power to Congress. Radical Republicans 
        worried that the amendment was not ``self-executing,'' i.e., its 
        actual implementation would depend on the whims of Congress.
        \item Many argued the amendment was the civil rights act in new 
        clothes. The scope of the ``privileges and immunities'' clause was 
        contested along the same lines at the civil rights formula in the 
        earlier act. Many of these concerns were justified, as proponents 
        made comments that the amendment's language \emph{should} be construed 
        broadly to end racial discrimination in all forms.\footnote{Casebook 
        pp. 308--09.}
        \item \S\ explicitly overrules \emph{Dred Scott} by establishing 
        birthright citizenship for all people, blacks included.
    \end{enumerate}
    \item Congress rejected other proposed amendments that explicitly required 
    color blindness. Does this mean that Congress did \emph{not} intend the 
    Equal Protection Clause to require color blindness in all situations?
\end{enumerate}

\subsubsection{The Fourteenth Amendment Limited}

\begin{enumerate}
    \item The Reconstruction amendments mainly aimed to end slavery, but they 
    also established protections for a broader range of rights. A key question 
    was, what rights did the new amendments guarantee?
    \item The right to vote was understood to be outside the scope of civil 
    rights, but others were strongly contested.
    \item The \emph{Slaughter-House Cases} tested the meaning of ``free labor'' 
    as a civil right.
    \item At the time, the Bill of Rights was understood to apply only to the 
    federal government. The states were free from its restrictions.
    \item \S\ 1 of the Thirteenth and Fourteenth Amendments are \textbf{self 
    executing} in that people can base claims on them alone without 
    additional statutory authority.
\end{enumerate}

\subsubsection{How Does the Court Decide What The Constitution Means? \emph{The 
Slaughter-House Cases}}

Is free labor a fundamental right? Do the Reconstruction amendments---the 
Fourteenth in particular---protect the right to labor, or does the power 
remain with the states?

The Court narrowly interpreted the privileges and immunities Clause in the 
Fourteenth Amendment. But some rights (e.g., the Bill of Rights) have been 
held to be protected against encroachment by states via the Due Process 
Clause.

\begin{enumerate}
    \item 1869: Louisiana wanted to move unpleasant slaughterhouses out of the 
    New Orleans city limits. It granted exclusive slaughterhouse rights to a 
    new corporation and required all New Orleans butchers to use its 
    facilities.  \item Justice Miller:
    \begin{enumerate}
        \item The butchers were not deprived of labor because they were free 
        to use the newly incorporated slaughterhouse.
        \item The state's police power authorized it to protect ``the general 
        interests of the community.''\footnote{Casebook p. 320.} Regulations of 
        the meat industry ``are among the most necessary and frequent exercise 
        of this power.''\footnote{Casebook p. 321.}
        \item The state has the power to incorporate a municipality, so it 
        should also have the power to form another corporation. The court did 
        not think that the monopoly privileges granted to this corporation 
        were ``especially odious or objectionable.''\footnote{Casebook p. 
        321.} Louisiana had this power unless something changed with the 
        adoption of the Reconstruction amendments.\footnote{Casebook p. 321.}
        \item \textbf{Butchers' Argument 1}: the statute created involuntary 
        servitude in violation of the Thirteenth Amendment.
        \begin{enumerate}
            \item The Reconstruction amendments were meant to end slavery. 
            The amendments have nothing to do with labor and property rights.
            \item The Court here asserted the power to discover the intent of 
            the amendment's authors based on its own experience of the 
            historical context.
        \end{enumerate}
        \item \textbf{Butchers' Argument 2}: the statute abridged privileges and 
        immunities of United States citizens.
        \begin{enumerate}
            \item The Fourteenth Amendment draws a clear distinction between 
            United States citizenship and state citizenship.
            \item The privileges and immunities clause protects only 
            \emph{federal} rights against encroachment by the 
            states.\footnote{``No State shall make or enforce any law which 
            shall abridge the privileges or immunities of citizens of the 
            United States~.~.~.~'' U.S.  Const. amend. XIV, \S\ 1.}
            \item \emph{Corfield v. Coryell}: ``privileges and immunities'' 
            covers a broad range of ``fundamental'' rights (property, 
            happiness, safety).\footnote{Casebook p. 324.} This broadness 
            would grant the Court the power to strike down a wide range of 
            state legislation, which would be an unacceptable structural 
            shift.
            \item States are responsible for a large domain of civil rights. 
            The Fourteenth Amendment did not expand the power of the Court to 
            act as a ``perpetual censor upon all legislation of the States, on 
            the civil rights of their own citizens~.~.~.~''\footnote{Casebook 
            p. 324.} Such an interpretation would also grant broad police 
            powers to Congress (in violation of the principle of enumerated 
            powers), which would ``fetter and degrade'' the states and 
            ``radically change[] the whole theory of the relations of the 
            State and Federal governments to each 
            other~.~.~.~''\footnote{Casebook p. 325.}
            \item Federal civil rights include asserting claims against the 
            government, right to assembly, right to petition for redress of 
            grievances, and habeus corpus.
        \end{enumerate}
        \item \textbf{Butchers' Argument 3}: the statute deprived butchers of property 
        without due process.
        \begin{enumerate}
            \item This was not a deprivation of property. The due process 
            argument is irrelevant.\footnote{Casebook p. 325--26.}
        \end{enumerate}
        \item \textbf{Butchers' Argument 4}: the statute denied equal protection.
        \begin{enumerate}
            \item No. The equal protection clause applies only to 
            slavery.\footnote{Casebook p. 326.}
        \end{enumerate}
    \end{enumerate}
    \item Justice Field, dissenting:
    \begin{enumerate}
        \item Police powers do not allow a state to violate constitutional 
        rights.
        \item Granting monopoly privileges to the new slaughterhouse 
        corporation did not promote the health of the city.
        \item Granting monopoly privileges to the government (e.g., ferries, 
        bridges) is different from granting monopoly privileges to private 
        entity in ``one of the ordinary trades or callings of 
        life.''\footnote{Casebook p. 326.} By the majority's rationale, 
        ``there is no monopoly, in the most odious form, which may not be 
        upheld.''\footnote{Casebook p. 327.}
        \item The Fourteenth Amendment \emph{does} protect U.S. citizens 
        against violations by state legislatures of ``common 
        rights'' (i.e., inalienable natural rights, like the right to property 
        and to pursue happiness)\footnote{Casebook p. 327.}
        \item Field made an anti-discrimination argument. He worried about 
        ``special and partial legislation'' would grant unfair privileges to 
        those who had the wealth to secure benefits for themselves.
    \end{enumerate}
    \item Justice Bradley, dissenting:
    \begin{enumerate}
        \item Blackstone identified three fundamental rights: personal 
        security, personal liberty, and private property. The freedom to choose a 
        profession is essential to the exercise of these 
        rights.\footnote{Casebook p. 328.}
        \item The original Constitution prevented the federal government from 
        abridging fundamental rights, which were enumerated in the text of the 
        Constitution and in the Bill of Rights. Now, the Fourteenth Amendment 
        extended the same protections to abridgement by the states.
        \item Ending slavery was not the sole purpose of the Reconstruction 
        amendments. ``It is futile to argue that none but persons of the 
        African race are intended to be benefited by this amendment. They may 
        have been the primary cause of the amendment, but its language is 
        general, embracing all citizens, and I think it was purposely so 
        expressed.''\footnote{Casebook p. 329.}
    \end{enumerate}
    \item Justice Swayne, dissenting:
    \begin{enumerate}
        \item The first eleven amendments put ``checks and limitations'' on 
        the federal government. The Reconstruction amendments limit state 
        power.
        \item ``Labor is property, and as such merits 
        protection~.~.~.~''\footnote{Casebook p. 330.}
        \item The Fourteenth Amendment granted the federal government power to 
        protect citizens' fundamental rights against oppression by the states. 
        This power is ``eminently conservative~.~.~.~bulwark of 
        defense~.~.~.~can never be made an engine of oppression~.~.~.~cannot 
        be abused.''\footnote{Casebook p. 330.}
        \item States are sometimes (often?) more oppressive than the federal 
        government---contrary to the founders' fears.
        \begin{enumerate}
            \item Swayne develops a political theory on the basis of national 
            experience. Cf. the ``new federalism'' after 1995.
        \end{enumerate}
    \end{enumerate}
    \item The belief in a ``general constitutional law''---based on some kind 
    of essential, natural law---allowed the Court to limit the power of both 
    the federal government and state governments, even if the legislation in 
    question did not explicitly violate the federal 
    Constitution.\footnote{Casebook p.  331.} Justices Field and Bradley 
    invoked this idea in arguing that the Fourteenth Amendment protected a set 
    of general, fundamental rights. But the Court later held that it did not 
    have jurisdiction to impose general constitutional law on the states in 
    cases heard on appeal from state courts.
    \item ``Due process'' in the Fifth and Fourteenth Amendments is not 
    defined. The Court earlier held that that ``due process'' means (1) 
    consistent with the Constitution and (2) consistent with common law 
    practice in England.\footnote{Casebook p. 333.}
    % TODO Citizenship: contrast with _Dred Scott_. p. 334 n. 7.
    \item Today, the \emph{Slaughter-House} interpretation of the Privileges 
    and Immunities Clause---that the protected rights are narrow range of 
    federal rights\footnote{Casebook p. 325}---is still good law. But most of 
    the Bill of Rights have been selectively incorporated into the Due Process 
    Clause. Thus, \emph{Slaughter-House} did not stop the rise of federal 
    police powers.
\end{enumerate}

\subsubsection{Early Application of the Fourteenth Amendment to Women} 

\paragraph{Women's Citizenship in the Antebellum Period}

\begin{enumerate}
    \item Suffrage was limited to propertied white males on the theory that 
    ``people should not be able to vote unless they possessed sufficient 
    independence to exercise the franchise wisely.'' Dependent people (women, 
    servants, etc.) would merely vote in the interests of their 
    masters.\footnote{Casebook pp. 164--65.}
    \item \textbf{Coverture}: upon marriage, a woman's legal rights are 
    subsumed into her husband's. Marital unity was a legal fiction under which 
    the husband and wife were one.
    \item Political rights were based on social status. People subordinate in 
    society and family---e.g., women---lacked status and thus lacked political 
    rights, including the right to vote.
    \item Status-based rights were justified on the basis of a distinction 
    between the \textbf{public realm} (economics, politics) and the 
    \textbf{private realm} (home, family).\footnote{Casebook p. 165.} 
    Moreover, women were already ``virtually represented'' by the male heads 
    of their households (husbands or fathers).\footnote{Casebook p. 168.}
    \item Another problem resulting from coverture: if a woman could not enter 
    into a private contract, how could she enter into the social 
    contract?\footnote{Casebook p. 166.}
    \item \emph{Shanks v. DuPont}: women do not automatically lose their 
    citizenship by marrying aliens (but in this case, Shanks did renounce her 
    American citizenship in favor of British citizenship). This was the first 
    case to recognize that women had any form of political 
    rights.\footnote{Casebook p. 166.}
    \item The Declaration of Sentiments from Seneca Falls demanded suffrage 
    and reform of marital status laws.
    \item A few states had passed ``married women's property acts'' and 
    ``earnings statutes,'' but these ``preserved the doctrine of marital 
    service.''\footnote{Casebook p. 167.}
\end{enumerate}

\paragraph{The Right to Practice Law: \emph{Bradwell v. Illinois}}
~\\\\
Is the right to choose a profession one of the privileges and immunities 
guaranteed to citizens of the United States?

\begin{enumerate}
    \item Illinois denied Myra Bradwell the right to practice law solely 
    because she was a woman. She argued that the privileges and immunities 
    Clause protected the rights in the Declaration of Independence. As part of 
    the pursuit of happiness, she was guaranteed the right to pursue the 
    profession of her choice. The protection of this right, she argued, should 
    be equal before the law.
    \item Justice Miller:
    \begin{enumerate}
        \item The right to practice law does not depend on United States 
        citizenship.
        \item \emph{Slaughter-House}: the Fourteenth Amendment did not 
        transfer to the federal government the right to regulate licenses to 
        practice law. This power remains with the states. (This case was 
        decided on the same day as \emph{Slaughter-House}.)
    \end{enumerate}
    \item Justice Bradley, concurring:
    \begin{enumerate}
        \item Bradley relies heavily on natural law.
        \item Bradwell assumes that women should be granted the same 
        privileges and immunities as men. But no, ``the civil law, as well as 
        nature herself, has always recognized a wide difference in the 
        respective spheres and destinies of man and woman.'' Women belong in 
        the domestic sphere. Their husbands represent their 
        interests.\footnote{Casebook p. 338.}
        \item As for unmarried women, they lack civil rights because they're 
        really supposed to be mothers anyway.
        \item Bradley's dissent relies not on an \textbf{authority-based 
        theory}, in which husbands exercise dominion over their wives, but on 
        an \textbf{affect-based theory}, in which women lack rights because of 
        their ``obligations to home and hearth.''\footnote{Casebook p. 339.}
    \end{enumerate}
\end{enumerate}

\paragraph{The ``New Departure'' and Women's Place in the Constitutional 
Order}

\begin{enumerate}
    \item The suffrage movement lost the battle to guarantee women's suffrage 
    in the Fourteenth Amendment. Stanton and Anthony opposed the Fifteenth 
    Amendment because of its ``humiliat[ing] rejection of extending suffrage 
    to women.''\footnote{Casebook p. 340.}
    \item Women's rights activists (e.g., Francis and Virginia Minor) began to 
    argue that suffrage \emph{is} a basic right of United States citizenship 
    and that the Fourteenth Amendment banned states from denying it. This 
    argument would be tested in \emph{Minor} (below).
    \item Also, it was established that naturalized citizens were guaranteed 
    the right to vote. The government could therefore not logically deny the 
    right to natural born citizens.
    \item Anthony: the Fourteenth Amendment automatically gave all citizens 
    the right to vote. Moreover, women were previously in a condition of 
    servitude to the males in their households, so the Fifteenth Amendment 
    prevented states from denying them the right to vote.
    \item 
\end{enumerate}

\paragraph{Does Citizenship Confer Suffrage? \emph{Minor v. Happersett}}

\begin{enumerate}
    \item Virginia Minor tried to register to vote in Missouri. After being 
    refused, she sued, arguing that women had a constitutional right to vote. 
    The Supreme Court unanimously rejected her argument.
    \item Justice Waite:
    \begin{enumerate}
        \item Are all citizens necessarily voters?
        \item The Fourteenth Amendment did not create new privileges or 
        immunities. It only furnished additional protections for existing 
        rights.
        \item Structural arguments:
        \begin{enumerate}
            \item When the Constitution was adopted, no state (except maybe 
            New Jersey) allowed all citizens to vote, including many explicit 
            restrictions on women's suffrage. Nothing in the Constitution 
            indicates intent to change these circumstances. If there were, 
            ``the framers~.~.~.~would not have left it to 
            implication.''\footnote{Casebook pp. 343--344.}
            \item If voting were a ``privilege,'' then voters would be allowed 
            to vote in multiple states, and that would be absurd.
            \item The language of the Fourteenth Amendment would not be 
            restricted to males if non-males had a right to vote (\S\ 2: males 
            can be disenfranchised, but states' representation will be reduced 
            proportionally).
            \item The Fifteenth Amendment would have been superfluous if 
            voting were a guaranteed privilege, because the guarantee would 
            have already been in place. Moreover, it does not provide gender 
            protections.
        \end{enumerate}
        \item Contextual arguments:
        \begin{enumerate}
            \item Women's suffrage has never been a requirement for admitting 
            a state to the Union.
            \item Non-citizens are sometimes allowed to vote. Therefore, the 
            right to vote does not depend on citizenship.
        \end{enumerate}
        \item ``Certainly, if the courts can consider any question settled, it 
        is this one.''\footnote{Casebook p. 345.}
    \end{enumerate}
\end{enumerate}
 
% \subsubsection{The Private Sphere and State Action} 
% 
% \paragraph{\emph{The Civil Rights Cases}}
% 
% \begin{enumerate}
%     \item % TODO 373-85
% \end{enumerate}
% 
\subsubsection{``Separate but Equal''}

\paragraph{Establishment of the ``Separate but Equal'' Doctrine}

\begin{enumerate}
    \item Many of the Court's post--Civil War decisions appear more 
    responsive to the idea of black inferiority than racial equality.
    \item \textbf{Compromise of 1877}: southern Democrats abandoned their 
    support for Samuel Tilden and supported Republican Rutherford B. Hayes in 
    exchange for the end of Reconstruction.
    \item C. Vann Woodward: the result of the compromise was a relaxation of 
    northern liberal opposition to southern segregationist policies. ``Just as 
    the Negro gained his emancipation and new rights through a falling out 
    between white men, he now stood to lose his rights through the 
    reconciliation of white men.''\footnote{Casebook p. 358.}
    \item \emph{The Civil Rights Cases} held that equal protection applies to 
    state actors, but not private actors. \emph{Plessy} held that equal 
    protection \emph{applied to} but did not \emph{prohibit} the Louisiana 
    statute.
    \item Congress based later civil rights legislation on its power to 
    regulate interstate commerce.
\end{enumerate}

\paragraph{\emph{Plessy v. Ferguson}}


\begin{enumerate}
    \item Homer Plessy challenged the constitutionality of a Louisiana law 
    requiring blacks and whites to ride in separate (but ``equal'') railroad 
    cars.
    \item \emph{Plessy} was a collusive lawsuit. The railroad didn't want to 
    bear the expense of establishing separate but equal facilities.
    \item Justice Brown:
    \begin{enumerate}
        \item This law did not violate the Thirteenth Amendment. That 
        amendment abolished servitude and only applied to labor systems. A 
        statute drawing ``merely a legal distinction'' between races ``has no 
        tendency to destroy the legal equality of the two races, or 
        reestablish a state of involuntary 
        certitude~.~.~.~''\footnote{Casebook p. 359. The Thirteenth Amendment 
        today is still interpreted in similarly narrow terms, ignoring the 
        possibility that group degradation is necessary to forced labor.}
        \item The Fourteenth Amendment aimed to abolish \emph{legal} 
        distinctions between races, but not \emph{social} distinctions (e.g., 
        marriage, education).  ``Separate but equal'' does not imply racial 
        inferiority. Nor does it abridge privileges and immunities, deprive 
        property without due process, or violate equal 
        protection.\footnote{Casebook pp. 359--60.}
        \item However, the act's of exemption from liability from damages is 
        unconstitutional.  
        \item The government's power to determine a passenger's race is not in 
        question here.
        \item The plaintiff argued that the reputation of being a member of the 
        dominant race is a form of property. The court agreed, but since blacks 
        were not the dominant race, they had no property interests at stake in 
        this case.
        \item The plaintiff also makes a slippery slope argument: why not 
        segregate on the basis of hair color, etc.? The court answered that 
        ``every exercise of police power must be reasonable, and extend only to 
        such laws as are enacted in good faith for the promotion of the public 
        good, and not for the annoyance or oppression of a particular 
        class.''\footnote{Casebook p. 361.}
        \item Was the Louisiana separate-but-equal statute unreasonable or 
        obnoxious? No.
        \item The plaintiff's ``underlying fallacy'' was that separate but equal 
        facilities ``stamps the colored race with a badge of inferiority.'' The 
        plaintiff's position assumed (wrongly) that if the racial roles were 
        reversed that whites would be oppressed. It also assumed (wrongly) that 
        social prejudices can be overcome through legislation.\footnote{Casebook 
        p. 361.} Law didn't cause group inequality and law can't fix it.
        \item The question of ``the proportion of colored blood necessary to 
        constitute a colored person'' was not at issue here.
    \end{enumerate}
    \item Justice Harlan, dissenting:
    \begin{enumerate}
        \item The Reconstruction amendments were meant to remove race as a 
        category on which the government could discriminate.
        \item The Thirteenth Amendment struck down ``badges of slavery or 
        servitude,'' of which separate but equal facilities is 
        one.\footnote{Casebook p. 362.}
        \item The Louisiana statute was clearly intended to keep blacks out of 
        whites' spaces. ``No one would be so wanting in candor as to assert the 
        contrary.''\footnote{Casebook p. 363.}
        \item Justice Harlan also made the slippery slope argument. But he 
        rejected the response exercise of the police power must be 
        ``reasonable,'' because courts do not have the power to assess the 
        ``policy or expediency of legislation.''\footnote{Casebook p. 363.}
        \item ``Our Constitution is color-blind, and neither knows nor 
        tolerates classes among the citizens.''\footnote{Casebook p. 363.} 
        \item The \emph{purpose} of the Reconstruction amendments was to grant 
        citizenship to and protect the privileges and immunities of blacks.
        \item The \emph{consequences} of disregarding this purpose would be to 
        ``permit the seeds of race hate to be planted under the sanction of 
        law.''\footnote{Casebook p. 364.} Law \emph{does} cause group 
        inequality.
        \item We denied citizenship to the Chinese, but allowed them to ride in 
        the same railroad cars as whites. How can we grant citizenship to 
        blacks, but make them ride in separate cars?
        \item ``The thin disguise of `equal' accommodations for passengers in 
        railroad coaches will not mislead any one, nor atone for the wrong this 
        day done.''\footnote{Casebook p. 365.}
        \item Though Harlan did not draw this distinction, modern commentators 
        have recognized two competing definitions of ``equality'':
        \begin{enumerate}
            \item \textbf{Anticlassification}: race is automatically suspect. 
            ``Our Constitution is color-blind.''
            \item \textbf{Antisubordination}: groups can be mistreated or 
            subordinated even when the law is racially neutral, e.g., poll 
            taxes.
            \item Both of these views would strike down \emph{Plessy}, but 
            critically, they differ on affirmative action.
            \item Harlan was skeptical of subordination that excluded groups 
            from \emph{markets}, but he did not have a problem with 
            \emph{social} subordination.
        \end{enumerate}
        % TODO discussion p. 365 ff.
    \end{enumerate}
\end{enumerate}

% \paragraph{The Spirit of \emph{Plessy}}
% 
% \begin{enumerate}
%     \item % TODO 370-73
% \end{enumerate}
% 
\subsection{Economic Rights and Structural Concerns}

\subsubsection{The \emph{Lochner} Era: Substantive Due Process} 

\paragraph{Pressures for Intervention and the Rise of Substantive Due 
Process, 1874--1890}

\begin{enumerate}
    \item By 1890, the Court had largely embraced Justice Bradley's dissent in 
    \emph{The Slaughter-House Cases} (the Fourteenth Amendment protects 
    fundamental rights against state encroachment).
    \item After \emph{Slaughter-House}, corporate lawyers found little aid in 
    the privileges and immunities clause, so they turned to due process.
    \item \emph{In Matter of Jacobs}: the New York Court of Appeals struck down a 
    statute prohibiting the manufacture of cigars in tenement houses on the 
    ground that it deprived owners and renters of property and 
    liberty.\footnote{Casebook pp. 412--13.}
    \item \emph{Godcharles v. Wigeman}: the Pennsylvania Supreme Court struck 
    down a law requiring companies to pay wages in cash instead of company 
    vouchers on the ground that it interfered with freedom of 
    contract.\footnote{Casebook p. 413.}
    \item Sui juris: people who possess full legal rights and capacity.
    \item \emph{Munn v. Illinois}: the Supreme Court upheld a law limiting 
    grain-storage warehouse charges. The source of the state's power was the 
    principle that allows regulation of private property when it is ``affected 
    with a public interest'' or ``affects the community at 
    large.''\footnote{Casebook p. 413.}
    \item \emph{The Railroad Commission Cases}: the Court upheld state 
    regulation of railroad tariffs, but cautioned that there are limits on 
    states' regulatory power.\footnote{Casebook p. 414.}
    \item \emph{Minnesota Rate Cases}: the Court ``struck down a statute 
    granting a state railroad commission unreviewable authority to set 
    rates'' on the principle that such regulations acted as deprivation of 
    property without due process. It was significant because (1) it held that 
    administrative rate-setting must be subject to judicial oversight and (2) 
    courts had the responsibility to determine whether established rates were 
    reasonable. It ``practically overrule[d] \emph{Munn v. Illinois}.'' The 
    Court went on to review regulations of almost every 
    kind.\footnote{Casebook pp. 414--15.}
    \item The \emph{Lochner} majority viewed the state as a fundamental threat 
    to liberty, espousing \emph{classic liberalism}. Justice Harlan's argued 
    that liberty requires freedom from severe constraints, following 
    \emph{modern liberalism}.
\end{enumerate}

\paragraph{Substantive Due Process: \emph{Lochner v. New York}}
~\\\\
The key tension in \emph{Lochner} was between constitutional liberty and 
police powers. When is state regulation an appropriate exercise of police 
power, and when is it ``an unreasonable, unnecessary, and arbitrary 
interference with the right of the individual to his personal liberty or to 
enter into those contracts in relation to labor which may seem to him 
appropriate or necessary for the support of himself and his 
family?''\footnote{Casebook p. 418.}

Today, almost everyone thinks \emph{Lochner} was wrong. Why?

\begin{enumerate}
    \item New York passed a statute limiting bakery employees to work no more 
    than sixty hours per week or ten hours per day. Lochner was convicted of 
    employing a baker for more than 60 hours per week.
    \item Justice Peckham:
    \begin{enumerate}
        \item The New York statute interfered with the right of contract 
        between employer and employee.
        \item ``The right to purchase or to sell labor is part of the liberty 
        protected by this amendment [the Fourteenth], unless there are 
        circumstances which exclude the right.''\footnote{Casebook p. 417.} 
        Liberty is not merely freedom from physical restraint.
        \item States can restrict certain kinds of contracts, like contracts 
        in violation of a statute or to let property for immoral purposes or 
        other unlawful purposes.
        \item When states regulate the rights of laborers, ``it becomes of 
        great importance to determine which shall prevail---the right of the 
        individual to labor for such time as he may choose, or the right of 
        the State to prevent the individual from laboring or from entering 
        into any contract to labor beyond a certain time prescribed by the 
        State.''\footnote{Casebook p. 417.}
        \item The Court previously upheld the regulation of labor contracts in 
        \emph{Holden v. Hardy} (involving miners)---but the Court dismisses 
        that precedent as irrelevant.
        \item This law was about the health of individuals working as bakers.
        \item To be valid, the regulation must have a clear and direct 
        connection to public health. ``Is this a fair, reasonable, and 
        appropriate exercise of the police power of the State, or is it an 
        unreasonable, unnecessary and arbitrary interference with the right of 
        the individual to his personal liberty or to enter into those 
        contracts in relation to labor which may seem to him appropriate or 
        necessary for the support of himself and his 
        family?''\footnote{Casebook p. 418.} There is no such connection here. 
        Baking is not inherently unhealthy, and such regulations ``cripple the 
        ability of the laborer to support himself and his 
        family.''\footnote{Casebook p. 419.}
        \item The court here increased the federal government's power by 
        asserting the power to review everything the states do in exercising 
        their police powers.
        \item The court never finds a due process violation. ``This is not a 
        question of substituting the judgment of this 
        court~.~.~.~''\footnote{Casebook p. 418.}---but in fact the court 
        \emph{does} substitute its judgment.
        \item If the principle is to keep the populace healthy and robust, 
        then the state would have the power to regulate the exertion of all 
        citizens---for instance, athletes. This argument resembles Justice 
        Marshall's slippery slope argument in \emph{Marbury}: if a statute is 
        passed by Congress, it would be de facto constitutional---an absurd 
        result.
        \item The New York legislature in passing the statute worried about 
        employees' lack of bargaining power. The court dismissed this 
        rationale, seemingly unwilling to allow legislatures to correct these 
        kinds of imbalances. But sometimes this kind of paternalism is 
        justified, according to the court---for instance, to secure ``real 
        equality of right'' for women.\footnote{Casebook p. 426.}
        \item Defendant also argued that longer hours led to unclean bread. 
        The Court disagreed.
        \item ``It seems to us that the real object and purpose were simply to 
        regulate the hours of labor between the master and his employ\'{e}s 
        (all being men, sui juris), in a private business, not dangerous in 
        any degree to morals or in any real and substantial degree, to the 
        health of the employees.''\footnote{Casebook p. 420.}
    \end{enumerate}
    \item Justice Harlan, dissenting:
    \begin{enumerate}
        \item When can courts strike down legislation? \emph{Only} ``if a 
        statute purporting to have been enacted to protect the public health, 
        the public morals or the public safety, has no real substantial 
        relation to those objects, or is beyond all question, a plain, 
        palpable invasion of rights secured by the fundamental 
        law.''\footnote{Casebook p. 420.} \item Baking is hazardous to your 
        health. This statute had a legitimate connection to public welfare.
        \item The Court's role is to defer to legislatures except in 
        extraordinary cases. The majority, on the other hand, held that courts 
        should review the reasonableness of legislation de novo. The majority 
        clearly distrusted legislatures---for instance, by calling the statute 
        in question ``mere pretext.''\footnote{Casebook p. 420.}
    \end{enumerate}
    \item Justice Holmes, dissenting:
    \begin{enumerate}
        \item
        \item The majority's opinion rests on an economic theory (presumably, 
        laissez faire and social Darwinism). ``[A] constitution is not 
        intended to embody a particular economic 
        theory~.~.~.~''\footnote{Casebook p. 422.}
        \item ``I think that the word liberty in the Fourteenth Amendment is 
        perverted when it is held to prevent the natural outcome of a dominant 
        opinion [as expressed through the legislature], unless it can be said 
        that a rational and fair man necessarily would admit that the statute 
        proposed would infringe fundamental 
        principles~.~.~.~''\footnote{Casebook p. 422.}
    \end{enumerate}
\end{enumerate}

\paragraph{The Transformation and Federalization of General Constitutional Law}

\begin{enumerate}
    \item \emph{Lochner} federalized \enquote{the principles of 
    \enquote{general constitutional law}}, which had previously only been 
    available in diversity cases.
    \item Under the Marshall and Taney Courts, the police power was thought to 
    be broad in scope. \emph{Lochner} signaled a restriction and the 
    separation of two distinct domains: individual autonomy and police 
    power. The Court policed the boundary between the two.\footnote{Casebook 
    pp. 422--23.}
    % TODO: vested rights doctrine under Marshall/Taney? see p. 422 bottom
\end{enumerate}

\paragraph{The Meanings of ``Liberty,'' ``Property,'' and ``Process''}

\begin{enumerate}
    \item \emph{Liberty}: ``The `liberty' mentioned in [the Fourteenth] 
    amendment means, not only the right of the citizen to be free from the 
    mere physical restraint of his person, as by incarceration, but the term 
    is deemed to embrace the right of the citizen to be free in the enjoyment 
    of all his faculties; to be free to use them in all lawful ways; to live 
    and work where he will; to earn his livelihood by any lawful calling; to 
    pursue any livelihood or avocation; and for that purpose to enter into all 
    contracts which may be proper, necessary, and essential to his carrying 
    out to a successful conclusion the purposes above 
    mentioned.''\footnote{\emph{Allgeyer v. Louisiana}---see casebook p. 423.}
    \begin{enumerate}
        \item Some criticized this definition on the ground that under English 
        common law and at the time the Constitution was written, ``liberty'' 
        was understood to relate solely to liberty of the person, i.e., 
        freedom from physical restraint.
    \end{enumerate}
    \item \emph{Property}: includes the right to make 
    contracts.\footnote{Casebook p. 424.}
    \item \emph{Process}: unclear. Legislative process? Judicial process?
\end{enumerate}

\paragraph{The Scope of the Police Power: Permissible and Impermissible 
Objectives}

\begin{enumerate}
    \item When can states exercise the police power? For what objectives?
    \item \emph{Holden v. Hardy}: in \emph{Lochner}, the Court distinguished 
    this case on the ground that the statute in question applied to miners, 
    not bakers. The Court found the \emph{Lochner} statute problematic 
    because it regulated labor without an obvious connection to public 
    welfare.
    \item \emph{Baltimore \& Ohio R. Co. v. Interstate Comm. Comm'n}: the 
    court upheld limitations on the hours of railroad employees in the name of 
    the safety of employees and travelers.\footnote{Casebook p. 424.}
    \item \emph{Coppage v. Kansas}: statutes that compensate for employees' 
    lack of bargaining power do not affect public welfare and therefore must 
    be struck down.\footnote{Casebook p. 425.}
    \item \emph{Muller v. Oregon}: legislation protecting women in the 
    workplace is valid because the state needs healthy 
    mothers.\footnote{Casebook p. 426.}
\end{enumerate}

\paragraph{Burdens of Proof and Questions of Degree}

\begin{enumerate}
    \item \emph{Lochner} (Peckham, majority): ``There [must] be some fair 
    ground, reasonable in and of itself, to say that there is a material 
    danger to the public health, or to the health of the employee, if the 
    hours of labor are not curtailed.''\footnote{Casebook p. 419.}
    \item \emph{Lochner} (Harlan, dissenting): law must ``have a real or 
    substantial relation'' to the promotion of public 
    health.\footnote{Casebook p. 420.}
    \item Peckham and Harlan appear to agree about the standard. Why did they 
    reach different results?
    \begin{enumerate}
        \item Maybe they apply the same standard with a different burden of 
        proof.
        \item Maybe Peckham categorizes occupations as inherently hazardous or 
        not, while Harlan puts them on a spectrum.
    \end{enumerate}
\end{enumerate}

\paragraph{Laissez Faire, Lawyers, and Legal Scholarship}

\begin{enumerate}
    \item Smith, Spencer, Sumner.
    \item Classic liberalism developed in reaction to authoritarianism in 
    government and economics. In an industrialized society, regulation may be 
    necessary to protect ``individual freedom and equality of 
    opportunity.''\footnote{Casebook pp. 428--29.}
\end{enumerate}

\paragraph{A Survey of the Court's Work}

\begin{enumerate}
    \item 1890--1934: the court struck down around 200 regulations, but 
    sustained as many, and declined to hear many more.
    \item ``The Court let stand most laws that appeared to protect the health, 
    safety, or morals of the general public or to prevent consumer 
    deception.''\footnote{Casebook p. 430.} It also permitted regulation of 
    rates for railroads and public utilities.
    \item Regulations like minimum wage laws ``seemed obviously designed to 
    readjust the market in favor of one party to the contract---and this was 
    entirely at odds with the underlying principle of laissez 
    faire.''\footnote{Casebook p. 431.}
\end{enumerate}

\subsubsection{The Commerce Clause} 

\paragraph{Congressional Regulation of Interstate Commerce}

\begin{enumerate}
    \item Before the Civil War, most cases addressing the commerce clause 
    involved the validity of state regulation. After the War, the focus 
    shifted to the scope of Congress's legislative powers.
    \item Congress began seriously regulating interstate trade with the 
    Interstate Commerce Act of 1887 and the Sherman Antitrust Act of 1890.
    \item The Court consistently upheld railroad 
    regulations.\footnote{Casebook pp. 435--36.}
    \item Three recurring doctrinal issues:\footnote{Casebook pp. 436--37.}
    \begin{enumerate}
        \item Whether the \emph{subject} of congressional regulation was 
        actually ``interstate commerce'' or some other local activity.
        \item Whether the \emph{purposes} of congressional regulation were 
        consistent with the purposes of the commerce clause.
        \item Whether the regulation ran afoul of the Tenth Amendment.
    \end{enumerate}
\end{enumerate}

\paragraph{\emph{Champion v. Ames}}
~\\\\
Congress can use its commerce clause powers to protect commerce as well as 
promote public policy goals.

\begin{enumerate}
    \item An 1895 congressional statute prohibited sending lottery tickets 
    between states through the mail. Champion, indicted for violating the law, 
    challenged its constitutionality.
    \item Justice Harlan:
    \begin{enumerate}
        \item Champion argued that mailing lottery tickets did not 
        constitute commerce. The government (Ames) argued that it did.
        \item The Constitution does not define ``commerce,'' but 
        ``undoubtedly'' it includes traffic in things with ``a recognized 
        value in money.''\footnote{Casebook p. 437.} Lottery tickets have 
        value.
        \item Champion next argued that Congress had the power to 
        \emph{regulate} but not \emph{prohibit} interstate commerce. The Court 
        rejected this argument, holding that even if Congress intended to 
        prohibit ``the widespread pestilence of lotteries,'' its power ``to 
        regulate commerce among the States is plenary, complete in itself, and 
        is subject to no limitations except such as may be found in the 
        Constitution.''\footnote{Casebook p. 438.}
        \item Champion argued further that the act violated the Tenth 
        Amendment. Held: no---the act does not limit commerce within the 
        states, only between them.
        \item ``~.~.~.~Congress, for the purpose of guarding the people of the 
        United States against the `widespread pestilence of lotteries' and to 
        protect the commerce which concerns all the States, may prohibit the 
        carrying of lottery tickets from one State to 
        another.''\footnote{Casebook p. 439.}
    \end{enumerate}
    \item Justice Fuller, dissenting:
    \begin{enumerate}
        \item Police power is reserved to the states. This act was an exercise 
        of police power. It was beyond Congress's authority in violation of 
        the Tenth Amendment.
        \item Lottery tickets are not ``articles of commerce'' any more than 
        private insurance contracts, which the Court had previously held to be 
        beyond the reach of Congress's authority to regulate under the 
        commerce clause. This act cannot be valid ``unless the power to 
        regulate interstate commerce includes the absolute and exclusive power 
        to prohibit the transportation of anything or anybody from one State 
        to another~.~.~.~''\footnote{Casebook pp. 439--40.}
        \item ``An invitation to dine, or to take a drive, or a note of 
        introduction, all become articles of commerce under the ruling in this 
        case, by being deposited with an express company for 
        transportation.''\footnote{Casebook p. 440.}
        % TODO: answer questions on p. 441
    \end{enumerate}
\end{enumerate}

\paragraph{\emph{Hammer v. Dagenhart}}

The power to prohibit commerce depends on the harmful nature of the prohibited 
trade. Lotteries, for instance, are a ``widespread pestilence'' and therefore 
fall within Congress's regulatory authority. \emph{Champion}. But child labor 
does not. Moreover, Congress does not have the authority to prevent unfair 
competition between states.

\begin{enumerate}
    \item Dagenhart worked in a cotton mill with his two minor sons. He 
    challenged the constitutionality of an act of Congress ``intended to 
    prevent interstate commerce in the products of child 
    labor.''\footnote{Casebook p. 441.}
    \item The district court held the act unconstitutional.
    \item Justice Day:
    \begin{enumerate}
        \item Key question: ``Is it within the authority of Congress in 
        regulating commerce among the States to prohibit the transportation in 
        interstate commerce of manufactured goods'' produced with child 
        labor?\footnote{Casebook p. 442.}
        \item The government (Hammer, the DA for the Western District of North 
        Carolina) argued that Congress was within its commerce clause 
        authority. The government argued that the commerce clause automatically 
        gives Congress to prohibit interstate trade of ``ordinary 
        commodities.'' The court disagreed, holding that the power to prohibit 
        depends on ``the character of the particular subjects dealt 
        with.''\footnote{Casebook p. 442.} For instance, in \emph{Champion}, 
        Congress's authority rested on the harmful effects of lotteries. But 
        ``[t]his element is wanting in the present case.''\footnote{Casebook 
        p. 442.}
        \item The government argued further that regulation was necessary to 
        prevent states that allowed child labor to gain unfair trade 
        advantages over states with child labor restrictions. The Court held 
        that Congress has no power ``to require the States to exercise their 
        police power so as to prevent possible unfair competition.'' Under any 
        other result, ``all freedom of commerce will be at an end, and the 
        power of the States over local matters may be eliminated and thus our 
        system of government be practically destroyed.''\footnote{Casebook p. 
        443.}
    \end{enumerate}
    \item Justice Holmes, dissenting:
    \begin{enumerate}
        \item Congress has the power to regulate (and therefore prohibit) 
        interstate commerce. Whether its regulations indirectly affect 
        economic activity within the states is irrelevant to the exercise of 
        its authority.
        \item Congress has the power---not the courts---to implement public 
        policy. ``It may carry out its views of public policy whatever 
        indirect effect they may have upon the activities of the 
        States.~.~.~.~The public policy of the United States is shaped with a 
        view to the benefit of the nation as a whole.''\footnote{Casebook pp. 
        444--45.}
        % TODO questions p. 445
    \end{enumerate}
\end{enumerate}

\paragraph{Prisonner's Dilemmas}

\begin{enumerate}
    \item State economies face a classic prisoner's dilemma: on their own, 
    states will be better off if they cooperate in economic policy, but each 
    has a stronger incentive to defect. So if state A passes a child labor 
    law, its production costs will rise, and state B will gain an economic 
    advantage---but both states would benefit more if they both passes child 
    labor laws.
    \item One counterargument is that this is not a true prisoner's dilemma 
    because states may not agree that child labor laws are best for their 
    citizens.
    \item The federal government can provide the necessary centralized 
    coordination to solve this dilemma.
\end{enumerate}
