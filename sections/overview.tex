\section{Overview}

\subsection{Separation of Powers, Federalism, and Reconstruction}

\begin{enumerate}
    \item Questions:
    \begin{enumerate}
        \item Which branch of government has the final say in constitutional 
        interpretation?
        \item Does the Bill of Rights apply to the states? How?
        \item What rights does the Fourteenth Amendment protect? Does it 
        create new rights?
        \item How would the Court today answer the butchers' due process and 
        equal protection claims in \emph{Slaughter-House}?
        \item Does ``separate but equal'' violate due process? Equal 
        protection?
        \item Would ``separate but equal'' violate due process or equal 
        protection if the Court had found that it \emph{does} stamp a ``badge 
        of inferiority''?
    \end{enumerate}
    \item \emph{Ware v. Hylton} (1796, Iredell): the Court assumed the power 
    to review state legislation under the Supremacy Clause.
    \item \emph{Marbury v. Madison} (1803, Marshall): the court held the 
    Judiciary Act of 1789 to be unconstitutional because in granting the Court 
    original jurisdiction in acts by government officials, Congress exceeded 
    its constitutional power.
    \begin{enumerate}
        \item \textbf{Judicial review}: the Court has the final say in 
        constitutional interpretation.  ``It is emphatically the province and 
        duty of the judicial department to say what the law 
        is.''\footnote{Casebook p. 116.}
    \end{enumerate}
    \item \emph{Dred Scott v. Sandford} (1857, Taney): Congress cannot ban 
    slavery in the states.
    \item \emph{The Slaughter-House Cases} (1873, Miller): the Privileges and 
    Immunities Clause forbade state infringement of the rights of 
    \emph{national} citizenship, but not the rights of state citizenship. A 
    broader interpretation would ``fetter and degrade'' the 
    states.\footnote{Casebook p. 325.} The Bill of Rights does not apply to 
    the states---at least not through Privileges and Immunities.
    \begin{enumerate}
        \item Field, dissenting: the Fourteenth Amendment \emph{does} protect 
        U.S. citizens against violations by state legislatures of ``common 
        rights'' (i.e., inalienable natural rights, like the right to property 
        and to pursue happiness)\footnote{Casebook p. 327.}
        \item The Fourteenth Amendment prevents states from infringing on the 
        liberty of \emph{all} people, not just blacks.
        \item The Fourteenth Amendment granted the federal government power to 
        protect citizens' fundamental rights against oppression by the states.
        \item Bradley, dissenting: the Fourteenth Amendment guaranteed the 
        butchers the liberty of employment---an early version of substantive 
        due process.
    \end{enumerate}
    \item \emph{Bradwell v. Illinois} (1873, Miller): the Fourteenth Amendment 
    did not transfer the power to regulate professional licenses to the 
    federal government. It remains with the states.
    \begin{enumerate}
        \item Bradley, concurring: women belong in the domestic sphere. Women 
        lack rights because of their ``obligations to home and 
        hearth.''\footnote{Casebook p. 339.}
    \end{enumerate}
    \item \emph{Minor v. Happersett} (1875, Waite): citizenship does not 
    confer suffrage. The Fourteenth Amendment did not create new privileges or 
    immunities. It only furnished additional protections for existing rights.
    \item \emph{Plessy v. Ferguson} (1896, Brown): the Fourteenth Amendment 
    abolished \emph{legal} but not \emph{social} distinctions between races.  
    ``Separate but equal'' does not impose a ``badge of 
    inferiority.''\footnote{Casebook p. 361.}
    \begin{enumerate}
        \item Harlan, dissenting: ``Our Constitution is color-blind, 
        and neither knows nor tolerates classes among the 
        citizens.''\footnote{Casebook p. 363.}
    \end{enumerate}
\end{enumerate}

\subsection{\emph{Lochner} and Substantive Due Process}

\begin{enumerate}
    \item Questions:
    \begin{enumerate}
        \item Is there a principled way to reconcile the \emph{Lochner}-era 
        opinions that appear inconsistent (e.g., upholding regulations 
        protecting coal miners but not bakers)?
        \item Should the Court avoid endorsing an economic theory? Or was the 
        problem with \emph{Lochner} that it endorsed the wrong theory?
        \item How does Lochnerism relate to classic liberalism?
        \item Why did the Court find that lottery tickets are harmful and thus 
        within Congress's power to regulate under the Commerce Clause 
        (\emph{Champion}), but goods produced with child labor are not 
        (\emph{Hammer})?
        \item Should Congress's Commerce Clause authority depend on whether 
        the regulated activity is ``harmful''? Who should decide what's 
        ``harmful''?
        \item How did the \emph{Lochner}-era court define ``liberty''?
    \end{enumerate}
    \item \emph{Lochner v. New York} (1905, Peckham): New York cannot limit 
    the number of hours a baker can work because the regulation violates the 
    Due Process Clause's protection of liberty.
    \begin{enumerate}
        \item Harlan, dissenting: when states pass legislation aimed at 
        protecting public health and safety, the Court should strike it down 
        only if it ``has \textbf{no real substantial relation} to those 
        objects, or is beyond all question, a plain, palpable invasion of 
        rights secured by the fundamental law.''\footnote{Casebook p. 420.} 
        This was an early version of rational review---see \emph{Carolene} 
        below.
        \item Holmes, dissenting:
        \begin{enumerate}
            \item The majority's opinion rests on an economic theory 
            (presumably, laissez faire and social Darwinism). ``[A] 
            constitution is not intended to embody a particular economic 
            theory~.~.~.~''\footnote{Casebook p. 422.}
            \item ``I think that the word liberty in the Fourteenth Amendment 
            is perverted when it is held to prevent the natural outcome of a 
            dominant opinion [as expressed through the legislature], unless it 
            can be said that a rational and fair man necessarily would admit 
            that the statute proposed would infringe fundamental 
            principles~.~.~.~''\footnote{Casebook p. 422.}
        \end{enumerate}
        \item \textbf{Lochnerism}: the Court strikes down regulation as 
        infringing on economic liberty, endorsing laissez-faire economics and 
        limiting states' power to enact social welfare legislation.
        \item Problems with Lochnerism:
        \begin{enumerate}
            \item Is the freedom to contract a fundamental right?
            \item The \emph{Lochner}-era Court was inconsistent---e.g., 
            upholding regulations for coal miners but not for bakers.
            \item The Court substituted its own values for those of 
            legislatures.
        \end{enumerate}
    \end{enumerate}
    \item \emph{Champion v. Ames} (1903, Harlan): Congress can regulate interstate 
    mailing of lottery tickets. Congress can use its Commerce Clause powers to 
    protect commerce as well as promote public policy goals.
    \begin{enumerate}
        \item Fuller, dissenting: lottery tickets are not ``articles of 
        commerce.'' Congress should not exercise police powers via the 
        Commerce Clause.
    \end{enumerate}
    \item \emph{Hammer v. Dagenhart} (1918): Congress can use its Commerce 
    Clause power only to regulate harmful activity. Lottery tickets are 
    harmful, but child labor is not.
    \begin{enumerate}
        \item Congress has the power to regulate (and therefore prohibit) 
        interstate commerce. Whether its regulations indirectly affect 
        economic activity within the states is irrelevant to the exercise of 
        its authority.
    \end{enumerate}
\end{enumerate}

\subsection{Economic Due Process}

\begin{enumerate}
    \item Questions:
    \begin{enumerate}
        \item What is ``economic'' due process? How does it differ from 
        substantive and procedural due process? % TODO see p. 309, 339 of UC
        \item How did the Court determine which parts of the Bill of Rights apply 
        against the states?
        \item Was Roosevelt's court packing scheme legitimate?
        \item What was the \emph{Carolene} Court's justification for substituting 
        rational review for \emph{Lochner}'s heightened review?
        \item Doesn't a minimum wage law (upheld in \emph{West Coast Hotel}) 
        restrict liberty in the same way as a maximum hours-per-week law (struck 
        down in \emph{Lochner})? How did the Court's understanding of liberty 
        change from \emph{Lochner} to \emph{Carolene}? Did the \emph{Carolene} 
        court disagree with \emph{Lochner}'s reasoning or its political 
        outcome (or both)?
        \item Is Footnote Four dictum?
        \item In \emph{Lee Optical}, why didn't the Court consider opticians a 
        discrete and insular minority?
    \end{enumerate}
    \item After the Civil War, the Bill of Rights increasingly applied to the 
    states via the Fourteenth Amendment.
    \item \emph{West Coast Hotel v. Parrish} (1937, Hughes): upholding a state 
    minimum wage law, the Court held that ``regulation which is reasonable in 
    relation to its subject and is adopted in the interests of the community 
    is due process.''\footnote{Casebook p. 516.}
    \item \emph{United States v. Carolene Products} (1938, Stone): the Court 
    upheld a prohibition on interstate commerce in filled milk. The Court 
    should presume the constitutionality of regulation unless it does not 
    ``rest upon some \textbf{rational basis}.''\footnote{Casebook p. 515.} 
    \begin{enumerate}
        \item \textbf{Footnote four}: the Court should apply heightened 
        scrutiny to provisions that (1) facially violate constitutional 
        provisions, (2) distorts political processes, or (3) affects discrete 
        and insular minorities.\footnote{Casebook p. 515.}
    \end{enumerate}
    \item \emph{Williamson v. Lee Optical} (1955): the Court upheld 
    regulations requiring licenses to do optical work. The Court will uphold 
    regulation if it can imagine a rational purpose, regardless of legislative 
    intent.
\end{enumerate}

\subsection{Expansion of the Commerce Clause}

\begin{enumerate}
    \item Questions:
    \begin{enumerate}
        \item How does the rationale in \emph{Wickard} differ from the 
        rationale in \emph{NFIB v. Sebelius}?
        \item Is it possible to reconcile \emph{Hammer} with the post-1937 
        Commerce Clause cases?
        \item Does \emph{Darby} mean that Congress is free to regulate 
        interstate commerce for noneconomic reasons? Or does it mean that 
        courts should not try to discern Congress's motives as long as the 
        regulation is a valid exercise of its commerce clause 
        power?\footnote{Casebook pp. 554--55.}
        \item Does the Commerce Clause give Congress general police powers? Is 
        that legitimate?
        \item How do the theories of Ackerman (``constitutional moments'') and 
        Levinson and Balkin (``partisan entrenchment'') explain the New Deal 
        shifts in the Court's application of the Commerce Clause and the Due 
        Process Clause?
    \end{enumerate}
    \item \emph{NLRB v. Jones \& Laughlin} (1937, Hughes): companies affect 
    interstate commerce when they interfere with organizing and collective 
    bargaining. ``Although activities may be intrastate in character when 
    separately considered, if they have such a \textbf{close and substantial 
    relation} to interstate commerce that their control is essential or 
    appropriate to protect that commerce from burdens and obstructions, 
    Congress cannot be denied the power to exercise that 
    control.''\footnote{Casebook p. 550.}
    \item \emph{United States v. Darby} (1941, Stone): Congress's freedom to 
    exercise its Commerce Clause powers do not depend whether ``its exercise 
    is attended by the same incidents which attend the exercise of the police 
    power of the states.''\footnote{Casebook p. 551.} Congress can also act as 
    a central coordinator to prevent unfair competition between states.
    \item \emph{Wickard v. Filburn} (1942, Jackson): ``~.~.~.~even if 
    appellee's activity be local and though it may not be regarded as 
    commerce, it may still, whatever its nature, be reached by Congress if it 
    exerts a \textbf{substantial economic effect on interstate 
    commerce}~.~.~.~''\footnote{Casebook pp. 553--54.}
\end{enumerate}

\subsection{Race}

\begin{enumerate}
    \item Questions:
    \begin{enumerate}
        \item To prove an equal protection violation, should you have to prove 
        discriminatory intent, disparate impact, or both? Can a strong enough 
        showing of discriminatory impact establish a prima facie case of 
        discriminatory intent (\emph{Yick Wo)}? Do people always intend the 
        ``natural consequences'' of their actions (Stevens concurring in 
        \emph{Davis})?
        \item Should the evidentiary standard for establishing discriminatory 
        intent depend on the context, e.g., education vs. employment?
        \item What was \emph{Brown}'s rationale? Is the holding legitimate 
        if the rationale is unclear (or missing)?
        \item If ``separate but equal'' facilities were truly equal, would 
        they still violate equal protection? Before \emph{Brown}, the NAACP 
        pushed to improve the quality of separate facilities. Is Warren right 
        that ``[s]eparate educational facilities are inherently unequal''?
        \item If the Southern Manifesto is right that the authors of the 
        Fourteenth Amendment did not intend it to affect segregation in 
        schools, would \emph{Brown} be wrong? Could \emph{Brown}'s positive 
        consequences (e.g., stigmatizing arguments in favor of segregation) 
        redeem it even if it contradicts the framers' intent?
        \item How does \emph{Brown} use social science? Does \emph{Brown}'s 
        use of social science contradict legal precedent?
        \item To what forms of racial segregation did \emph{Brown} apply? Did 
        its holding extend beyond education? ``\emph{Brown} did not proscribe 
        racial classification or declare it suspect. Rather, it 
        addressed the harmful consequences of separating school children in a 
        particular institutional context.''\footnote{Casebook p. 958.}
        \item Should the Court intervene to remedy resegregation resulting 
        from private action, like white flight (\emph{Milliken})?
        \item Could the Court have decided \emph{Brown} on due process 
        grounds?
        \item What's the relationship between equal protection and due process 
        in \emph{Loving}?
        \item Should we require legislators to confront the racial impact of 
        their policies (like environmental impact statements, which we already 
        require)?\footnote{Casebook p. 1034.}
        \item Why do racial quotas violate the Fourteenth Amendment?
        \item Does the \emph{Feeney} standard account for cognitive bias and 
        unconscious racism? Should it?
        \item When is statistical evidence of discriminatory impact strong 
        enough to show discriminatory purpose? Compare \emph{Yick Wo} (finding 
        a discriminatory purpose from statistical evidence showing that 
        Chinese laundromat operators were regularly denied permits) and 
        \emph{McCleskey} (finding no discriminatory purpose despite 
        statistical evidence that in Georgia, blacks were significantly more 
        likely than whites to receive the death penalty, especially when 
        killing whites).
        \item Under the current doctrine, could public universities implement 
        affirmative action programs based on gender? Or wealth, or alienage?
        \item Why not extend the goal of enhancing cross-racial understanding 
        to civil service jobs, or all jobs (Scalia in \emph{Grutter})?
    \end{enumerate}
    \item Three types of race cases:
    \begin{enumerate}
        \item \textbf{Jim Crow}: formal, overt racial distinctions.
        \item \textbf{Disproportionate impact}: no overt discrimination, but 
        the decision impacts certain groups more than their size warrants. How 
        should the law respond? Do you need to prove intent?
        \item \textbf{Affirmative action}: racial classification for remedial 
        purposes.
    \end{enumerate}
    \item \emph{Chae Chan Ping vs. United States} (1889, Field): there are no 
    limits on Congress's power to regulate immigration by discriminating on 
    the basis of race or national origin (though today, administrative 
    agencies cannot discriminate).
    \item \emph{Korematsu v. United States} (1944, Black): \textbf{strict 
    scrutiny} will invalidate a government's violation of the rights of a 
    \textbf{suspect class} unless a necessary government interest justifies 
    the violation.
    \item \emph{Brown v. Board of Education} (1954, Warren): ``~.~.~.~in the 
    field of public education the doctrine of `separate but equal' has no 
    place. Separate educational facilities are inherently 
    unequal.''\footnote{Casebook p. 902.}
    \begin{enumerate}
        \item \emph{Brown II} (1955, Warren): local courts are responsible for 
        implementing \emph{Brown}. Desegregation need not be immediate.
        \item The ``Southern Manifesto'': the framers of the Fourteenth 
        Amendment did not intend it to affect education. Indeed, the 
        Amendment's authors were the same men who voted to segregate 
        Washington, D.C.'s schools.
    \end{enumerate}
    \item \emph{Milliken v. Bradley}, (1974, Burger): for the first time since 
    \emph{Brown}, the Court found that a district court had gone too far in 
    remedying segregation. The lower court found that only one district had 
    intentionally discriminated, so the Supreme Court was unwilling to grant 
    an interdistrict remedy.
    \begin{enumerate}
        \item Justice White, dissenting: an interdistrict remedy would be more 
        effective under the circumstances than an intracity remedy.
        \item Justice Marshall, dissenting: the school board was acting as an 
        agent of the state. Since the state was responsible, an interdistrict 
        remedy should have been available.
        \item The Court later held that ``[w]here resegregation is a product 
        not of state action but of private choices, it does not have 
        constitutional implications.''\footnote{Casebook p. 945.} 
        Constitutional remedies are limited when white flight effects 
        resegregation.
    \end{enumerate}
    \item \emph{Hernandez v. Texas} (1954, Warren): the Court developed a 
    two-part test for identifying race-based equal protection violations.  
    First, is there a distinct class? Second, is there systematic 
    discrimination against that class?
    \begin{enumerate}
        \item This is similar to the rationale for \emph{Brown}, but here the 
        reasoning is explicit.
    \end{enumerate}
    \item \emph{Loving v. Virginia} (1967, Warren): when race is involved in 
    any way, the Court moves from rational review (``any legitimate state 
    interest'') to strict scrutiny (``compelling government interest''). Race 
    is intimately tied to group subordination.
    \begin{enumerate}
        \item \emph{Loving} also struck down the last pillar of Jim 
        Crow---antimiscegenation laws---and held that racial discrimination 
        implicates both equal protection (i.e., equality) and due process 
        (i.e., liberty).
    \end{enumerate}
    \item When should courts inquire into whether a decision was 
    race-dependent?
    \begin{enumerate}
        \item \emph{Yick Wo v. Hopkins}---Discriminatory 
        Administration of an Otherwise ``Neutral'' Statute: the San Francisco 
        Board of Supervisors granted laundromat permits to nearly all of 80 
        Caucasion applicants and none of 200 Chinese applicants. The Court 
        held that there was no reason for it except to express hostility to a 
        group.
        \item ``Queue Ordinance Case'': \emph{Ho Ah Kow v. 
        Nunan}---The Race-Dependent Decision to Adopt a Nonracially Specific 
        Regulation or Law: every male prisoner's hair had to be cut to within 
        one inch. The practical effect was to coerce Chinese people into 
        paying fines, because they dreaded the cutting of the queue (a braid 
        they held sacred).
        \begin{enumerate}
            \item \textbf{Gomillion v. Lightfoot}): the Alabama legislature 
            changed the boundaries of Tuskegee from a square to ``an uncouth 
            twenty-eight-sided figure.''\footnote{Casebook p. 1023.} The sole 
            purpose was to segregate voters by race.
        \end{enumerate}
        \item \emph{Gaston County v. United States}---Transferred De 
        Jure Discrimination: a voting literacy test disproportionately 
        disenfranchised blacks. The Court held that years of inferior 
        education meant that blacks were less equipped to pass the test, so 
        the test was discriminatory.
    \end{enumerate}
    \item \emph{Griggs v. Duke Power} (1971, Burger): Title VII prohibits 
    employers from requiring job applicants to have high school diplomas and 
    pass a general intelligence test without showing that those criteria 
    predict job performance. The Civil Rights Act ``proscribes not only overt 
    discrimination but also practices that are fair in form, but 
    \textbf{discriminatory in operation}.'' % TODO citation
    \item \emph{Washington v. Davis} (1976, White): the Court declined to read 
    the \emph{Griggs} ``disparate impact'' standard into the Fourteenth 
    Amendment. A law is not unconstitutional ``\emph{solely} because it has a 
    racially disproportionate impact.''\footnote{Casebook p. 1027.}
    \item \emph{Arlington Heights v. Metropolitan Housing Corp.} established 
    factors that courts can use to determine when government decisions are 
    racially motivated:
    \begin{enumerate}
        \item The impact of the action, including patterns that emerge.
        \item The decision's historical background.
        \item The sequence events leading up to the decision.
        \item ``Departures from the normal procedural 
        sequence.''\footnote{Casebook p. 1040.}
        \item Substantive departures from normal procedure.
        \item Legislative or administrative history.
    \end{enumerate}
    \item \emph{United Jewish Organizations of Williamsburg v. Carey} (1977, 
    White): states can base decisions on race as long as stigma is not 
    involved.
    \begin{enumerate}
        \item Stewart, concurring: racial awareness is not unconstitutional 
        per se. If there is no disparate impact (\emph{Davis}) and no 
        discriminatory intent, there is no constitutional violation.
        \item Brennan, concurring: benign racial classifications can have 
        invidious results---for instance, they can ``serve to stimulate our 
        society's latent race consciousness.''\footnote{Handout p. 3.}
    \end{enumerate}
    \item \emph{University of California v. Bakke} (1978, Powell): \emph{all} 
    racial classifications are suspect, even if benign. The Davis program had 
    four goals: (1) reduce the historic deficit of minorities in medical 
    school and the medical profession, (2) counter the effects of societal 
    discrimination, (3) increase the number of physicians who will work in 
    underserved communities, and (4) obtain the educational benefits that flow 
    from a \textbf{diverse student body}. The Court invalidated all but the 
    last.
    \begin{enumerate}
        \item Blackmun, concurring and dissenting: race-neutral affirmative 
        action is impossible. ``[I]n order to treat some persons equally, we 
        must treat them differently.''\footnote{Handout p. 8.}
        \item Marshall, concurring and dissenting: our long history of racial 
        inequality created a range of present inequalities. The Fourteenth 
        Amendment does not prohibit remedies for past discrimination.
    \end{enumerate}
    \item \emph{Richmond v. Croson} (1989, O'Connor): race-based 
    classifications are subject to strict scrutiny.  Race-based affirmative 
    action programs must be related to a compelling government interest---the 
    same standard as programs that intentionally discriminate \emph{against} 
    racial groups.
    \item \emph{Adarand v. Pena} (1995, O'Connor): the \emph{Croson} 
    rule---that any racial classification is subject to strict 
    scrutiny---applies to Congressional actions as well as state and local 
    actions. 
    \item \emph{Personnel Administrator of Massachusetts v. Feeney} (1979, 
    Stewart): the Court elaborated the \emph{Davis} discriminatory purpose 
    requirement, holding that foreseeable impact ``was not sufficient to prove 
    discriminatory purpose under the Equal Protection Clause'':
    \begin{enumerate}
        \item \enquote{\enquote{Discriminatory purpose}~.~.~.~implies more 
        than intent as volition or intent as awareness of consequences. It 
        implies that the decisionmaker, in this case a state legislature, 
        selected or reaffirmed a particular course of action at least in part 
        \textbf{`because of,' not merely `in spite of,' its adverse effects 
        upon an identifiable group.}}\footnote{Casebook p. 1031.}
    \end{enumerate}
    \item \emph{McCleskey v. Kemp} (1987, Powell): statistical evidence of 
    discriminatory impact (at least in this case) is insufficient to prove
    discriminatory intent. 
    \begin{enumerate}
        \item Brennan, dissenting: race was a major factor in McCleskey's 
        death sentence. Racism remains.
        \item Blackmun, dissenting: the statistical evidence showed that 
        McCleskey was a member of a distinct class singled out for different 
        treatment. The burden should have shifted to the sttae to rebut 
        evidence of discriminatory intent.
    \end{enumerate}
    \item \emph{Grutter v. Bolinger} (2003, O'Connor): following  
    \emph{Bakke}, the court held that any use of race warrants strict 
    scrutiny, which requires a compelling governmental interest. Student body 
    diversity is such an interest. The affirmative action program must also be 
    narrowly tailored.
    \begin{enumerate}
        \item Rehnquist, dissenting: this program was not narrowly tailored, 
        and it operated like a quota system.
        \item Kennedy, dissenting: the school has not shown that individual 
        assessment was safeguarded throughout the process. The Court is too 
        deferential.
        \item Scalia, dissenting: why not extend the goal of enhancing 
        cross-racial understanding to civil service jobs, or all jobs?
        \item Thomas, dissenting: racial diversity is not a compelling 
        government interest.
    \end{enumerate}
    \item \emph{Gratz v. Bollinger} (2003, Rehnquist): universities using 
    point systems for admissions cannot grant automatic bonuses to minority 
    applicants because of their race.
    \item \emph{Parents Involved in Community Schools v. Seattle School 
    District No. 1} (2007, Roberts): schools cannot consider race when 
    assigning students to schools, even if racial imbalances exist between 
    schools.
    \item \emph{Ricci v. DeStefano} (2009, Kennedy): % FIXME 
\end{enumerate}

\subsection{Gender}

\begin{enumerate}
    \item Questions:
    \begin{enumerate}
        \item How would \emph{Bradwell} be decided today?
        \item \emph{Reed} was an equal protection case. \emph{Frontiero} was a 
        due process case. How do the Court's analyses differ?
        \item Catherine MacKinnon: ``Socially, one tells a woman from a man by 
        their difference from each other, but a woman is legally recognized to 
        be discriminated against on the basis of sex only when she can first 
        be said to be the same as a man.''\footnote{Casebook p. 1211.} How 
        does this apply to \emph{Frontiero}?
        \item Why does the Court apply different standards of review for race 
        and gender classifications?
        \item What was wrong with Brennan's rationale in \emph{Frontiero} for 
        applying heightened scrutiny to gender classifications?
        \item Why do we allow ``separate but equal'' facilities for different 
        genders but not different races?
        \item Does Ginsburg's ``exceedingly persuasive justification'' 
        standard of review in \emph{VMI} differ from the standards for gender 
        classifications in earlier cases?
    \end{enumerate}
    \item \emph{Bradwell v. Illinois} (1873, Bradley, concurring): the Court 
    rejected an early attempt to apply the Fourteenth Amendment to sex 
    discrimination. Similarly, the right to vote is not a privilege or 
    immunity.\footnote{Casebook p. 1180.}
    \item \emph{Adkins v. Children's Hospital} (1923, Sutherland): a minimum 
    wage law for women violated freedom of contract. Overruled in \emph{West 
    Coast Hotel} (1937).
    \item \emph{Reed v. Reed} (Burger, 1971): while purportedly applying 
    rational review, the Court struck down an Idaho law that preferred men 
    over women as estate administrators as ``the very kind of arbitrary 
    legislative choice forbidden by the Equal Protection Clause of the 
    Fourteenth Amendment.''\footnote{Casebook p. 1183.}
    \item \emph{Frontiero v. Richardson} (1973, Brennan): gender-based 
    classifications require heightened scrutiny. Intermediate scrutiny as the 
    standard won a plurality, not a majority. The majority adopted it in 
    \emph{Craig}.
    \item Race-gender analogies involve two strategies: (1) asking whether the 
    rationales for declaring race a suspect class also apply to gender, and 
    (2) identifying the features of race that make it special and asking 
    whether gender shares those features.
    \item There are important differences between race and gender---e.g., race 
    has a history of disdain and overt subordination, while gender has a 
    history of paternalism and less overt subordination.\footnote{Casebook pp. 
    1206--07.} See \S\ 4.3.2.1.
    \item \emph{Craig v. Boren} (1976, Brennan): the Court adopted 
    intermediate scrutiny for gender classifications, adopting the plurality 
    view from \emph{Frontiero}. ``To regulate in a sex-discriminatory fashion, 
    the government must demonstrate that its use of sex-based criteria is 
    \enquote{substantially related} to the achievement of \enquote{important 
    government objectives}.'' It never explained why it employed different 
    standards for race and gender.\footnote{Casebook p. 1214.}
    \item \emph{Baker v. State} (1999): the Vermont Supreme Court held that 
    sex-based marriage restrictions violated the state's constitution by 
    enforcing sex stereotypes without any legitimate reason.\footnote{Casebook 
    pp. 1221--22.}
    \item \emph{J.E.B. v. Alabama} (1994, Blackmun): gender-based peremptory 
    challenges are not allowed.
    \item \emph{Personnel Administrator of Massachusetts v. 
    Feeney} (1979, Stewart): if a statute has a disparate impact on one 
    gender, it is invalid only if the legislature passed it \emph{because of} 
    a desire to discriminate, not merely \emph{in spite of} knowledge that a 
    disparate impact would result.
    \item  \emph{Geduldig v. Aviello} (1974, Stewart): legislation that 
    differentiates between sexes based on biological factors is not 
    necessarily discriminatory. Applying the \emph{Feeney} principle, the 
    Court held that pregnancy was not a pretext for the legislature to 
    invidiously discriminate against women.
    \item \emph{Michael M. v. Superior Court of Sonoma} (1981, Rehnquist): 
    although they involve gender discrimination, statutory rape laws are 
    substantially related to the important government objective of preventing 
    teenage pregnancies.
    \item \emph{The VMI Cases: United States v. Virginia} (1996, Ginsburg): 
    single-gender education is permissible as long there are separate but 
    equal facilities for both males and females. Any gender-based state 
    action requires an ``exceedingly persuasive justification.''
    \begin{enumerate}
        \item Rehnquist, concurring: Ginsburg's standard of review departs 
        from earlier standards, introducing uncertainty. Also, ``it is not the 
        `exclusion of women' that violates the Equal Protection Clause, but 
        the maintenance of an all-men's school without providing any---much 
        less a comparable---institution for women.''\footnote{Casebook p. 
        1241.}
        \item Scalia, concurring: single-sex education can carry substantial 
        benefits, so it is ``substantially related'' to a state's educational 
        interests.\footnote{Casebook pp. 1243--44.}
    \end{enumerate}
\end{enumerate}

\subsection{Modern Substantive Due Process}

\begin{enumerate}
    \item Questions:
    \begin{enumerate}
        \item Douglas rejected Lochnerism in \emph{Griswold} (``[w]e do not 
        sit as a super-legislature~.~.~.~''\footnote{Casebook p. 1343}). But 
        is his analysis actually different from \emph{Lochner}?
        \item Why was \emph{Griswold} a due process case while 
        \emph{Eisenstadt} was based on equal protection?
        \item Is Ginsburg right that the Court should have waited to decide on 
        abortion because of \emph{Roe}'s political backlash?
        \item Could \emph{Roe} have been decided on equal protection grounds? 
        Should it have been?
        \item Why did the court decide \emph{Bowers} on due process 
        grounds but \emph{Lawrence} on equal protection grounds?
        \item Should sexual orientation be a suspect classification?
        \item Why is wealth not a suspect classification while alienage is? 
        Why do we recognize legal aliens as a suspect class (\emph{Graham}) 
        but not undocumented aliens (\emph{Plyler})?
    \end{enumerate}
    \item \emph{Griswold v. Connecticut} (1965, Douglas): the Court 
    invalidated a Connecticut statute outlawing contraceptive use. The 
    the Bill of Rights casts a ``penumbra'' that creates a ``zone of 
    privacy.''\footnote{Casebook p. 1343.} The Fourteenth Amendment protects 
    this liberty from state infringement.
    \begin{enumerate}
        \item Goldberg, concurring: marital privacy is implied in the Ninth 
        Amendment as a ``fundamental and basic'' right.
        \item Harlan, concurring: the liberty guaranteed by the Due Process 
        Clause is not limited to the precise terms of the rest of the 
        Constitution.  It protects \textbf{\enquote{basic values 
        \enquote{implicit in the concept of ordered liberty.}}}
        \item White, concurring: the statute \emph{could} withstand strict 
        scrutiny if it were substantially related to the stated goal (to deter 
        illicit relationships)---but it's not.
        \item Black, dissenting: there is no fundamental constitutional right 
        to privacy. The Court cannot reliably discern ``fundamental principles 
        of liberty and justice.''\footnote{Casebook p. 1351.}
    \end{enumerate}
    \item \emph{Eisenstadt v. Baird} (1972, Brennan): a statute banning 
    contraceptives for (among other things) preventing pregnancy for unmarried 
    couples violated the Equal Protection Clause because of its distinction 
    between married and unmarried couples. But the court avoided the question 
    of whether access to contraception is a fundamental right.
    \item \emph{Roe v. Wade} (1973, Blackmun): ``This right of privacy, 
    whether it be founded in the Fourteenth Amendment's conception of personal 
    liberty and restrictions upon state action, as we feel it is, or~.~.~.~in 
    the Ninth Amendment's reservations of rights to the people, is broad 
    enough to encompass a woman's decision whether or not to terminate her 
    pregnancy.'' Forcing a woman to continue a pregnancy imposes physical and 
    psychological burdens. Because abortion is a fundamental right, the 
    standard is strict scrutiny. The state has a compelling interest in 
    protecting the mother's health after the first trimester, and in 
    protecting prenatal life after the second trimester. But the first 
    trimester can only be regulated like any other medical procedure.
    \begin{enumerate}
        \item Justice Rehnquist, dissenting: % TODO
        \item Justice White, dissenting: % TODO
    \end{enumerate}
    \item (`abortion and the equal protection clause') % TODO
    \item (`decisions after roe') % TODO
    \item \emph{Planned Parenthood v. Casey} (1992, O'Connor): the Court 
    upheld the right to abortion but also upheld notice requirements.
    \item \emph{Gonzalez v. Carhart} (2007, Kennedy): the Court upheld the 
    Partial-Birth Abortion Ban Act, holding that the state had a substantial 
    interest in protecting fetal life.
    \item Sexual orientation and due process:
    \begin{enumerate}
        \item \emph{Bowers v.  Hardwick} (1986, White): the right to privacy 
        (under due process) does not protect the right to engage in private 
        homosexual activity. % TODO expand
    \end{enumerate}
    \item Sexual orientation and equal protection:
    \begin{enumerate}
        \item \emph{Romer v. Evans} (1996, Kennedy): there is no rational 
        basis for discriminating against homosexuals solely because of 
        animosity towards homosexuality. % TODO expand
        \item \emph{Lawrence v. Texas} (2003, Kennedy): states cannot prohibit 
        private sexual activity between consenting adults of the same sex. 
            % TODO expand
        \item (`Sexual Orientation as a Suspect Classification') % TODO
    \end{enumerate}
    \item Same-sex marriage:
    \begin{enumerate}
        \item (california marriage cases) % TODO
    \end{enumerate}
    \item \textbf{Other suspect classifications and fundamental rights}:
    \begin{enumerate}
        \item \emph{San Antonio Independent School District v. Rodriguez} 
        (1973, Powell): wealth (or poverty) is not a suspect classification 
        because it is large and amorphous. Education is an important right, 
        but not a constitutionally guaranteed right. ``Some inequality'' among 
        school districts is insufficient to warrant heightened review. The 
        Texas system passed rational review. % TODO expand
        \item (citizenship and alienage under the equal protection clause)
            % TODO
        \item \emph{Graham v. Richardson}, (1971, Blackmun): legal aliens are 
        a suspect class, calling for heightened scrutiny. The Court 
        invalidated an Arizona law that required citizenship or 15 years of 
        residence to receive welfare benefits.
        \item \emph{Bernal v. Fainter} (1984, Marshall): a Texas law required 
        notary publics to be US citizens. Alienage classifications warrant 
        suspect classifications. The \textbf{``political function'' exception} 
        allowed alienage discrimination for ``positions intimately related to 
        the process of democratic self-governance.'' The Court held that 
        notary publics did not qualify for the political function exception, 
        so it struck down the Texas law.
        \item (Regulation of Resident Aliens) % TODO
        \item \emph{Plyler v. Doe} (1982, Brennan): intermediate scrutiny 
        applies to laws affecting children because of their immigration 
        status.
    \end{enumerate}
\end{enumerate}

\subsection{Modern Commerce Clause}

\begin{enumerate}
    \item Questions:
    \begin{enumerate}
        \item Why did Congress enact civil rights legislation on the authority 
        of the Commerce Clause rather than the \S\ 2 of the Thirteenth 
        Amendment and \S\ 5 of the Fourteenth Amendment?
        \item Is Rehnquist in \emph{Lopez} correct that if the Court accepted 
        the government's arguments, then the federal government could 
        regulate all areas of criminal law?
        \item Does the modern definition of ``commerce'' go beyond the 
        framers' definition? Did the framers intend to bind later generations 
        to the same definition of commerce?
        \item Could the federal government tax people who don't buy broccoli?
        \item Is Scalia correct that abstention from commerce isn't 
        commerce?\footnote{\emph{NFIB v. Sebelius}, handout p. 18.} If so, how 
        can the federal government argue that it has the power to create 
        commerce where it doesn't exist?
    \end{enumerate}
    \item \emph{Heart of Atlanta Motel v. McClung} (1964, Clark): the Court 
    unanimously upheld Title II as a valid exercise of Congress's Commerce 
    Clause power, holding that the motel ``stood readily accessible to 
    interstate highways, advertised in various national media, and served a 
    clientele 75 percent of which came from out of state.''\footnote{Casebook 
    p. 560.}
    \item \emph{Katzenbach v. McClung} (1964, Clark): the Court held that 
    racial discrimination affects interstate commerce. It also held that 
    rational review is the appropriate standard for Commerce Clause cases.
    \item \emph{United States v. Lopez} (Rehnquist, 1995): early Commerce 
    Clause cases recognized Congress's authority to  to (1) regulate activity 
    that uses the \emph{channels} of interstate commerce, (2) protect the 
    \emph{instrumentalities} of interstate commerce, and (3) regulate 
    activities with a \emph{substantial relation} to interstate 
    commerce.\footnote{Casebook pp. 601--02.} The government argued that guns 
    have a substantial relation to interstate commerce. The Court disagreed, 
    holding that if the government's theories were correct, the federal 
    government could regulate all areas of criminal law under the Commerce 
    Clause.
    \begin{enumerate}
        \item Thomas, concurring: we should eliminate the ``substantial 
        relation'' test because it expands the definition of ``commerce'' 
        beyond the framers' intent.
        \item Stevens, dissenting: the welfare of commerce depends on 
        education. Guns threaten education, so they also threaten commerce.
        \item Souter, dissenting: the Court returns to pre-1937 \enquote{highly 
        formalistic notions of \enquote{commerce}} (e.g., \emph{Hammer}).
        \item Breyer, dissenting: gun violence has marked effects on 
        education, and in turn, on commerce.
    \end{enumerate}
    \item \emph{National Federation of Independent Businesses v. Sibelius} 
    (2012, Roberts): the Commerce Clause only gives Congress the power to 
    regulate economic \emph{activity}. It does not grant the power to 
    \emph{create} economic activity.\footnote{Handout p. 5.} However, the 
    individual mandate is valid under Congress's taxing power. ``~.~.~.~it 
    makes going without insurance just another thing the government 
    taxes.''\footnote{Handout p. 10.}
    \begin{enumerate}
        \item Ginsburg, concurring and  dissenting: the framers intended the 
        Constitution to be a \textbf{``great outline''} with the ``capacity to 
        provide for future contingencies as they may 
        happen~.~.~.''\footnote{Handout p.  11, quoting Hamilton in Federalist 
        No. 34.} Congress had a rational basis for concluding that the 
        uninsured substantially affected interstate commerce.
        \item Scalia, dissenting: Abstention from commerce is not 
        commerce.\footnote{Handout p.  18.} The individual mandate directs the 
        creation of commerce. There is not universal participation in the 
        healthcare market, and the federal government has no power to mandate 
        it.
    \end{enumerate}
\end{enumerate}

\subsection{Limits on the Fourteenth Amendment}
% TODO add dissents

\begin{enumerate}
    \item Questions:
    \begin{enumerate}
        \item Why did \emph{Jones} rely on the Thirteenth Amendment and not 
        the Fourteenth (either under equal protection or due process)?
        \item How does \emph{Boerne} help predict how the Court will rule in 
        \emph{Shelby County v. Holder}?
    \end{enumerate}
    \item \emph{Jones v. Alfred H. Mayer Co. } (1968, Stewart): Congress has the 
    power under the Thirteenth Amendment to prohibit racial discrimination in 
    the sale or rental of real estate (upholding the 1866 Civil Rights Act, 
    authorizing Congress to determine ``what are the badges and incidents of 
    slavery, and the authority to translate that determination into effective 
    legislation.'').
    \item \emph{Oregon v. Mitchell} (1970, Black): % TODO
    \begin{enumerate}
        \item Oregon v. Mitchell: was the only case between 1937 and 1987 to 
        hold an act of Congress unconstitutional based on a lack of enumerated 
        power. The Rehnquist Court signaled a shift towards limiting federal 
        powers.
    \end{enumerate}
    \item (`the reconstruction power') % TODO
    \item \emph{City of Boerne v. Flores} (1997, Kennedy): Congress lacks the 
    power ``to determine what constitutes a constitutional violation.''
        % TODO expand
    \item \emph{Northwest Austin Municipal Utility District Number One 
    (NAMUDNO) v. Holder} (2009, Roberts): \S\ 5 of the Voting Rights Act 
    requires pre-clearance for electoral law changes in certain districts. The 
    Court did not reach the question here, but strongly suggested that \S\ 5 
    exceeds Congress's power.
\end{enumerate}
