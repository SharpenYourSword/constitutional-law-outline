\section{Overview}

\subsection{Separation of Powers, Federalism, and Reconstruction}

\begin{enumerate}
    \item \emph{Ware v. Hylton} (1796, Iredell): the Court assumed the power 
    to review state legislation under the Supremacy Clause.
    \item \emph{Marbury v. Madison} (1803, Marshall): the court held the 
    Judiciary Act of 1789 to be unconstitutional because in granting the Court 
    original jurisdiction in acts by government officials, Congress exceeded 
    its constitutional power.
    \begin{enumerate}
        \item \textbf{Judicial review}: the Court has the final say in 
        constitutional interpretation.  ``It is emphatically the province and 
        duty of the judicial department to say what the law 
        is.''\footnote{Casebook p. 116.}
    \end{enumerate}
    \item \emph{Dred Scott v. Sandford} (1857, Taney): Congress cannot ban 
    slavery in the states.
    \item \emph{The Slaughter-House Cases} (1873, Miller): the Privileges and 
    Immunities Clause forbade state infringement of the rights of 
    \emph{national} citizenship, but not the rights of state citizenship. A 
    broader interpretation would ``fetter and degrade'' the 
    states.\footnote{Casebook p. 325.} The Bill of Rights does not apply to 
    the states---at least not through Privileges and Immunities.
    \begin{enumerate}
        \item Field, dissenting: the Fourteenth Amendment \emph{does} protect 
        U.S. citizens against violations by state legislatures of ``common 
        rights'' (i.e., inalienable natural rights, like the right to property 
        and to pursue happiness)\footnote{Casebook p. 327.}
        \item The Fourteenth Amendment prevents states from infringing on the 
        liberty of \emph{all} people, not just blacks.
        \item The Fourteenth Amendment granted the federal government power to 
        protect citizens' fundamental rights against oppression by the states.
    \end{enumerate}
    \item \emph{Bradwell v. Illinois} (1873, Miller): the Fourteenth Amendment 
    did not transfer the power to regulate professional licenses to the 
    federal government. It remains with the states.
    \begin{enumerate}
        \item Bradley, concurring: women belong in the domestic sphere. Women 
        lack rights because of their ``obligations to home and 
        hearth.''\footnote{Casebook p. 339.}
    \end{enumerate}
    \item \emph{Minor v. Happersett} (1875, Waite): citizenship does not 
    confer suffrage. The Fourteenth Amendment did not create new privileges or 
    immunities. It only furnished additional protections for existing rights.
    \item \emph{Plessy v. Ferguson} (1896, Brown): the Fourteenth Amendment 
    abolished \emph{legal} but not \emph{social} distinctions between races.  
    ``Separate but equal'' does not impose a ``badge of 
    inferiority.''\footnote{Casebook p. 361.}
    \begin{enumerate}
        \item Harlan, dissenting: ``Our Constitution is color-blind, 
        and neither knows nor tolerates classes among the 
        citizens.''\footnote{Casebook p. 363.}
    \end{enumerate}
\end{enumerate}

\subsection{\emph{Lochner} and Substantive Due Process}
% TODO add dissents

\begin{enumerate}
    \item \emph{Lochner v. New York} (1905): New York cannot limit the number 
    of hours a baker can work because the regulation violates the Due Process 
    Clause's protection of liberty.
    \begin{enumerate}
        \item \textbf{Lochnerism}: the Court strikes down regulation as 
        infringing on economic liberty.
        \item Problems with Lochnerism:
        \begin{enumerate}
            \item Is the freedom to contract a fundamental right?
            \item The \emph{Lochner}-era Court was inconsistent---e.g., 
            upholding regulations for coal miners but not for bakers.
            \item The Court substituted its own values for those of 
            legislatures.
        \end{enumerate}
    \end{enumerate}
    \item \emph{Champion v. Ames} (1903): Congress can regulate interstate 
    mailing of lottery tickets. Congress can use its Commerce Clause powers to 
    protect commerce as well as promote public policy goals.
    \item \emph{Hammer v. Dagenhart} (1918): Congress can use its Commerce 
    Clause power only to regulate harmful activity. Lottery tickets are 
    harmful, but child labor is not.
\end{enumerate}

\subsection{Economic Due Process}
% TODO add dissents

\begin{enumerate}
    \item After the Civil War, the Bill of Rights increasingly applied to the 
    states via the Fourteenth Amendment.
    % TODO: adkins v childrens hospital
    \item \emph{West Coast Hotel v. Parrish} (1937): minimum wage laws for 
    women are legitimate. % TODO expand
    \item \emph{United States v. Carolene Products} (1938): the Court upheld a 
    prohibition on interstate commerce in filled milk. The Court should 
    presume the constitutionality of regulation unless it does not ``rest upon 
    some \textbf{rational basis}.''\footnote{Casebook p. 515.} 
    \begin{enumerate}
        \item \textbf{Footnote four}: the Court should apply heightened 
        scrutiny to provisions that (1) facially violate constitutional 
        provisions, (2) distorts political processes, or (3) affects discrete 
        and insular minorities.\footnote{Casebook p. 515.}
    \end{enumerate}
    \item \emph{Williamson v. Lee Optical} (1955): the Court upheld 
    regulations requiring licenses to do optical work. The Court will uphold 
    regulation if it can imagine a rational purpose, regardless of legislative 
    intent.
\end{enumerate}

\subsection{Expansion of the Commerce Clause}
% TODO add dissents

\begin{enumerate}
    \item [jones \& laughlin, darby, wickard] % TODO
\end{enumerate}

\subsection{Race}
% TODO add dissents

\begin{enumerate}
    \item % TODO nat'l security: chan chae ping, korematsu (strict scrutiny)
    \item [Brown, brown II] % TODO
    \item The Court developed a two-part test for identifying race-based equal 
    protection violations. First, is there a distinct class? Second, is there 
    systematic discrimination against that class? \emph{Hernandez v. Texas}.
    \begin{enumerate}
        \item This is similar to the rationale for \emph{Brown}, but here the 
        reasoning is explicit.
    \end{enumerate}
    \item When race is involved in any way, the Court moves from rational 
    review (``any legitimate state interest'') to strict scrutiny 
    (``compelling government interest''). \emph{Loving v. Virginia}.
    \begin{enumerate}
        \item \emph{Loving} also struck down the last pillar of Jim 
        Crow---antimiscegenation laws---and held that racial discrimination 
        implicates both equal protection (i.e., equality) and due process 
        (i.e., liberty).
    \end{enumerate}
    \item (applying strict scrutiny to race: yick wo, etc.) % TODO
    \item Title VII prohibits employers from requiring job applicants to have 
    high school diplomas and pass a general intelligence test without showing 
    that those criteria predict job performance. \emph{Griggs v. Duke Power}.
    \item The Court declined to read the \emph{Griggs} ``disparate impact'' 
    standard into the Fourteenth Amendment. \emph{Washington v. Davis}.
    \item \emph{Arlington Heights v. Metropolitan Housing Corp.} established 
    factors that courts can use to determine when government decisions are 
    racially motivated:
    \begin{enumerate}
        \item The impact of the action, including patterns that emerge.
        \item The decision's historical background.
        \item The sequence events leading up to the decision.
        \item ``Departures from the normal procedural 
        sequence.''\footnote{Casebook p. 1040.}
        \item Substantive departures from normal procedure.
        \item Legislative or administrative history.
    \end{enumerate}
    % TODO UJO
    \item UC Davis's affirmative action plan was unconstitutional. However, 
    race is not categorically off limits. \emph{University of California v. 
    Bakke}. % TODO expand
    \item Race-based classifications are subject to strict scrutiny. 
    Race-based affirmative action programs must be related to a compelling 
    government interest---the same standard as programs that intentionally 
    discriminate \emph{against} racial groups. \emph{Richmond v. Croson}.
    \item The \emph{Croson} rule---that any racial classification is subject 
    to strict scrutiny---applies to Congressional actions as well as state and 
    local actions. \emph{Adarand v. Pena}.
    \item [McCleskey] % TODO
    \item [Grutter] % TODO
    \item [Gratz] % TODO
    \item [Ricci] % TODO
    \item Universities using point systems for admissions cannot grant 
    automatic bonuses to minority applicants because of their race. 
    \emph{Gratz v. Bollinger}.
    \item Schools cannot consider race when assigning students to schools, 
    even if racial imbalances exist between schools. \emph{Parents Involved in 
    Community Schools v. Seattle School District No. 1}.
\end{enumerate}

\subsection{Gender}
% TODO add dissents

\begin{enumerate}
    \item [early cases -- bradwell, adkins, reed] % TODO § 4.3.1.1
    \item Gender-based classifications require heightened scrutiny. 
    \emph{Frontiero v. Richardson}.
    \item Theory: see 'Relevant Differences or Stereotypes' % TODO
    \item [craig v. boren] % TODO
    \item \emph{Baker v. State}: the Vermont Supreme Court 
    held that sex-based marriage restrictions violated the state's 
    constitution by enforcing sex stereotypes without any legitimate 
    reason.\footnote{Casebook pp. 1221--22.}
    \item [j.e.b. v. alabama] % TODO
    \item If a statute has a disparate impact on one gender, it is invalid 
    only if the legislature passed it \emph{because of} a desire to 
    discriminate, not merely \emph{in spite of} knowledge that a disparate 
    impact would result. \emph{Personnel Administrator of Massachusetts v. 
    Feeney}.
    \item Legislation that differentiates between sexes based on biological 
    factors is not necessarily discriminatory. Applying the \emph{Feeney} 
    principle, the Court held that pregnancy was not a pretext for the 
    legislature to invidiously discriminate against women. \emph{Geduldig v. 
    Aviello}.
    \item [michael m.] % TODO
    \item [vmi] % TODO
\end{enumerate}

\subsection{Modern Substantive Due Process}
% TODO add dissents

\begin{enumerate}
    \item (`The Ninth Amendment') % FIXME
    \item ('Antecedents of Fundamental Rights Adjudication') % FIXME
    \item \emph{Griswold v. Connecticut} (1965, Douglas): the Court 
    invalidated a Connecticut statute outlawing contraceptive use. The 
    the Bill of Rights casts a ``penumbra'' that creates a ``zone of 
    privacy.''\footnote{Casebook p. 1343.}
    \begin{enumerate}
        \item Goldberg, concurring: marital privacy is implied in the Ninth 
        Amendment as a ``fundamental and basic'' right.
        \item Harlan, concurring: the liberty guaranteed by the Due Process 
        Clause is not limited to the precise terms of the rest of the 
        constitution.
        \item White, concurring: the statute \emph{could} withstand strict 
        scrutiny if it were substantially related to the stated goal (to deter 
        illicit relationships)---but it's not.
        \item Black, dissenting: there is no fundamental constitutional right 
        to privacy. The Court cannot reliably discern ``fundamental principles 
        of liberty and justice.''\footnote{Casebook p. 1351.}
    \end{enumerate}
    \item \emph{Eisenstadt v. Baird} (1972, Brennan): a statute banning 
    contraceptives for (among other things) preventing pregnancy for unmarried 
    couples violated the Equal Protection Clause because of its distinction 
    between married and unmarried couples. But the court avoided the question 
    of whether access to contraception is a fundamental right.
    \item (theory: see 'Theories of Fundamental Rights Adjudication') % FIXME
    \item \emph{Roe v. Wade} (1973, Blackmun): ``This right of privacy, 
    whether it be founded in the Fourteenth Amendment's conception of personal 
    liberty and restrictions upon state action, as we feel it is, or~.~.~.~in 
    the Ninth Amendment's reservations of rights to the people, is broad 
    enough to encompass a woman's decision whether or not to terminate her 
    pregnancy.'' % TODO cite
    Forcing a woman to continue a pregnancy imposes physical and psychological 
    burdens. Because abortion is a fundamental right, the standard is strict 
    scrutiny. The state has a compelling interest in protecting the mother's 
    health after the first trimester, and in protecting prenatal life after 
    the second trimester. But the first trimester can only be regulated like 
    any other medical procedure.
    \begin{enumerate}
        \item Justice Rehnquist, dissenting: % TODO
        \item Justice White, dissenting: % TODO
    \end{enumerate}
    \item (`abortion and the equal protection clause') % TODO
    \item (`decisions after roe') % TODO
    \item \emph{Planned Parenthood v. Casey} (1992, O'Connor): the Court 
    upheld the right to abortion but also upheld notice requirements.
        % TODO expand
    \item \emph{Gonzalez v. Carhart} (2007, Kennedy): the Court upheld the 
    Partial-Birth Abortion Ban Act, holding that the state had a substantial 
    interest in protecting fetal life. % TODO expand
    \item Sexual orientation and due process:
    \begin{enumerate}
        \item \emph{Bowers v.  Hardwick} (1986, White): the right to privacy 
        (under due process) does not protect the right to engage in private 
        homosexual activity. % TODO expand
    \end{enumerate}
    \item Sexual orientation and equal protection:
    \begin{enumerate}
        \item \emph{Romer v. Evans} (1996, Kennedy): there is no rational 
        basis for discriminating against homosexuals solely because of 
        animosity towards homosexuality. % TODO expand
        \item \emph{Lawrence v. Texas} (2003, Kennedy): states cannot prohibit 
        private sexual activity between consenting adults of the same sex. 
            % TODO expand
        \item (`Sexual Orientation as a Suspect Classification') % TODO
    \end{enumerate}
    \item Same-sex marriage:
    \begin{enumerate}
        \item (california marriage cases) % TODO
    \end{enumerate}
    \item \textbf{Other suspect classifications and fundamental rights}:
    \begin{enumerate}
        \item \emph{San Antonio Independent School District v. Rodriguez} 
        (1973, Powell): wealth (or poverty) is not a suspect classification 
        because it is large and amorphous. Education is an important right, 
        but not a constitutionally guaranteed right. ``Some inequality'' among 
        school districts is insufficient to warrant heightened review. The 
        Texas system passed rational review. % TODO expand
        \item (citizenship and alienage under the equal protection clause)
            % TODO
        \item \emph{Graham v. Richardson}, (1971, Blackmun): alienage is a 
        suspect classification, calling for heightened scrutiny. The Court 
        invalidated an Arizona law that required citizenship or 15 years of 
        residence to receive welfare benefits. % TODO expand
        \item \emph{Bernal v. Fainter} (1984, Marshall): a Texas law required 
        notary publics to be US citizens. Alienage classifications warrant 
        suspect classifications. The ``political function'' exception allowed 
        alienage discrimination for ``positions intimately related to the 
        process of democratic self-governance.'' The Court held that notary 
        publics did not qualify for the political function exception, so it 
        struck down the Texas law. % TODO expand
        \item (Regulation of Resident Aliens) % TODO
        \item \emph{Plyler v. Doe} (1982, Brennan): intermediate scrutiny 
        applies to laws affecting children because of their immigration 
        status. % TODO expand
    \end{enumerate}
\end{enumerate}

\subsection{Modern Commerce Clause}
% TODO add dissents

\begin{enumerate}
    \item (heart of atlanta motel and mcclung) % TODO
    \item (oregon v mitchell) % TODO
    \item (united states v lopez) % TODO
    \item (health care reform case) % TODO
\end{enumerate}

\subsection{Limits on the Fourteenth Amendment}
% TODO add dissents

\begin{enumerate}
    \item (jones v mayer, oregon v mitchell) % TODO
    \item (`the reconstruction power') % TODO
    \item (boerne v flores) % TODO
    \item (NAMUDNO) % TODO
\end{enumerate}
