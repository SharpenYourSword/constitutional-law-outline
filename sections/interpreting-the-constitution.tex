\section{Interpreting the Constitution}

\subsection{Background}

\begin{enumerate}
    \item The Articles of Confederation:
    \begin{enumerate}
        \item Submitted to the states in 1777; took effect in 1781 when the 
        last state, Maryland, assented.
        \item ``.~.~.~a treaty among a group of small nations~.~.~.~''\footnote{Casebook p. 
        19.}
        \item Voluntary taxes: ``pompous petitions for 
        charity.''\footnote{Casebook p. 19.}
    \end{enumerate}
    \item Feb. 1787: Congress authorized a convention ``for the sole and 
    express purpose of revising the Articles of 
    Confederation''---but it led to the drafting of the new Constitution. 
    \footnote{Casebook p. 21.}
    \item Amendments to the Articles of Confederation needed the states' 
    unanimous consent, but Article VII of the Constitution only required nine 
    states' approval. Rhode Island rejected it and North Carolina postponed 
    ratification. The Constitution took formal effect in June 21, 1788 when 
    the ninth state, New Hampshire, ratified. By Washington's inauguration in 
    1789, 11 states had ratified.
    \item \textbf{Necessary and Proper Clause}: ``The Congress shall have 
    Power~.~.~.~To make all Laws which shall be necessary and proper for carrying 
    into Execution the foregoing Powers.''\footnote{U.S. Const. art. I, \S\ 8; 
    casebook p. 4.}
    \item Should there have been a Bill of Rights?
    \begin{enumerate}
        \item In Federalist 84, Hamilton responded to calls for a bill of 
        rights: ``the bill of rights~.~.~.~would afford a colourable pretext 
        to claim more [powers] than were granted. For why declare that things 
        shall not be done which there is no power to do?~.~.~.~it is evident 
        it would furnish to men disposed to usurp, a plausible pretence for 
        claiming that power.''\footnote{Casebook p. 25.}
        \item Anti-federalist responses to Hamilton's argument pointed out 
        that the Constitution emphasizes limitations elsewhere, e.g., the 
        limitation on granting titles of nobility.\footnote{U.S. Const. art. 
        I, \S\ 9; casebook p. 26.}
    \end{enumerate}
    \item 1791: states ratified ten of twelve proposed amendments.
    \item \textbf{Federalists}: % TODO. Wanted to enhance federal power.
    Lost the election of 1800 and ``utterly collapsed following its failure to 
    adequately support the war of 1812.''\footnote{Casebook p. 137.}
    \item \textbf{Anti-Federalists}: % TODO
    \item \textbf{Democratic-Republicans}: % TODO
    \item \textbf{Separation of powers}: executive vs. judicial vs. 
    legislative.
    \item \textbf{Federalism}: federal government vs. states, i.e., limits on 
    federal power.
    \item \textbf{Political question doctrine}: the Supreme Court does not 
    resolve political questions. Those questions are resolved through the 
    political process.
    \item Most Supreme Court cases do not involve constitutional issues.
    \item States have general (but subordinate) \textbf{police powers}. The 
    federal government has limited (but supreme) \textbf{enumerated powers}. 
    In case of conflict, federal law preempts state law because of the 
    \textbf{Supremacy Clause}.\footnote{U.S. Const. art. VI, cl. 2.}
    \item \textbf{Preamble}: radical because authority comes from below; 
    flawed because structurally it supports slavery (e.g., 3/5ths vote).
\end{enumerate}

\subsection{Supreme Court as Expositor of the Constitution}

\subsubsection{Supreme Court, 1798--1801}

\begin{enumerate}
    \item ``Riding circuit'': there were no circuit courts, so Supreme Court 
    Justices traveled to sit with district judges.
    \item Early on, the Court assumed power to:
    \begin{enumerate}
        \item Review state legislation that conflicted with 
        federal treaties and statutes---e.g., \emph{Ware v. Hylton} (first 
        establishing judicial review of state law under the supremacy 
        clause).\footnote{Casebook p. 97.}
        \item ``~.~.~.~construe federal legislation in light of 
        presumably binding constitutional requirements.''\footnote{Casebook p. 
        97}
    \end{enumerate}
    \item ``~.~.~.~the basic assumption underlying \emph{Marbury} seems to 
    have been relatively well established by 1796.''\footnote{Casebook p. 98.}
    \item Justice Marshall aimed for a unanimous ``Opinion of the Court.''
    % TODO: after reading McCulloch, add material here about popular 
    % sovereignty -- see casebook p. 98, marginal note towards bottom, and 
    % sovereign immunity p. 99.
\end{enumerate}

\subsubsection{Election of 1800}

\begin{enumerate}
    \item Federalist No. 10: Madison warned of ``factions.'' The ``permanent 
    and aggregate interests of the community'' should outweigh the strategic 
    interests of a political party.
    \item Two parties emerged by the election of 1796: \textbf{Federalists} 
    (Adams, Hamilton) and \textbf{Democratic-Republicans} (Jefferson).
    \item The electoral college picked both the President and VP. In 1796, 
    Adams (a Federalist) got 71 votes and became President, and Jefferson (a 
    Democratic-Republican) got 68 and became VP.
    \item 1800:
    \begin{enumerate}
        \item Jefferson and his ``de facto running mate'' Burr both got 73 
        votes each. Adams received 65.
        \item The tie sent the election to the House, where each state's 
        delegation had one vote.\footnote{U.S. Const. art. II, \S\ 1, cl. 3.} 
        Although Jefferson's Republicans had won a majority in the next 
        Congress, the current lame-duck Congress still had a Federalist 
        majority. The Federalist ``irreconcilables'' tried to block Jefferson 
        and Burr's victory, until they finally gave up and allowed their 
        states to vote for Jefferson. Jefferson became President and Burr VP.
    \end{enumerate}
    \item The Twelfth Amendment (1804) separated the Presidential and VP 
    ballots, allowing both candidates to run on a ticket.
    \item The Federalist Congress in its last days passed the Judiciary Act of 
    1801, creating circuit courts and eliminating the practice of riding 
    circuit. The Republicans feared the new crop of judges would tilt the 
    courts in favor of the Federalists. The new Republican Congress repealed 
    it in 1802 and passed the Judiciary Act of 1802, reassigning Justices to 
    ride circuit. The Federalist raised two 
    objections: \begin{enumerate}
        \item Could the circuit judges legitimately be eliminated? The ``good 
        behavior'' clause presumably gave tenure for life.\footnote{U.S. 
        Const. art. III, \S\ 1.}
        \item Was it constitutional to assign Supreme Court justices to duty 
        on a lower court?
    \end{enumerate}
    \item Fearing a challenge to the constitutionality of the Repeal Act, the 
    Republicans eliminated the Supreme Court's 1802 term.
    \item ``...the central issue in both \emph{Stuart v. Laird} and 
    \emph{Marbury v. Madison} was whether the Court would directly challenge 
    th combined weight of executive and congressional authority, by carrying 
    out the implications of earlier decisions and actually invalidating a 
    federal statute, thereby potentially provoking a full-scale constitutional 
    crisis.''\footnote{Casebook p. 103.}
\end{enumerate}

\subsubsection{Judicial Review: \emph{Marbury v. Madison}}

Does the Court have the final word in constitutional interpretation? Both 
Congress and the executive branch also interpret the Constitution, but whose 
say is final?

\begin{enumerate}
    \item Adams had issued and signed an order appointing Marbury a justice of 
    the peace.  Marshall, then Secretary of State, failed to deliver it before 
    the end of the Adams administration. Jefferson and Madison, the new 
    Secretary of State, refused to deliver it. Marbury sued in December 1801 
    to receive the commission.
    \item The ``peculiar delicacy'' of the case was its contentious political 
    context and the impact of the Court's decision on the relations between 
    the branches of government. Madison's failure to appear in court resulted 
    from the controversy.
    \item Justice Marshall:
    \begin{enumerate}
        \item Did Marbury have a right to the commission?
        \begin{enumerate}
            \item Yes. The commission was complete when the President signed 
            it and the Secretary of State affixed the U.S. seal.
        \end{enumerate}
        \item If Marbury had a right to the commission, did he have right to a 
        legal remedy? When can courts review the executive's actions?
        \begin{enumerate}
            \item Yes. Acts ``in cases in which the executive possesses a 
            constitutional or legal discretion~.~.~.~are only politically 
            examinable''---i.e., they are beyond judicial review. But when the 
            act arises under a legal duty, ``and individual rights depend upon 
            the performance of that duty, it seems equally clear that the 
            individual who considers himself injured, has a right to resort to 
            the laws of his country for a remedy.~.~.~.''\footnote{Casebook p.  
            112.} 
            \item ``The government of the United States has been emphatically 
            termed a government of laws, and not of men.''\footnote{Casebook 
            p. 111.} Compare to Louis XIV: ``The State, it is I.''
            \item The question of whether an officer has a vested right in his 
            appointment, and therefore whether the President can remove him at 
            will, is for the courts. Therefore the courts can review Marbury's 
            dispute.
            \item Whether a right is vested is a question for the courts to 
            decide. Cf. the political question doctrine.
        \end{enumerate}
        \item If Marbury had a right to a remedy, was it a writ of mandamus 
        from the Supreme Court? Can the Court enforce the remedy? Can the 
        Court declare a statute to be unconstitutional? Why?
        \begin{enumerate}
            \item The Judiciary Act of 1789, which established the federal 
            court system, authorized the Supreme Court to issue mandamus to 
            lower courts or officials (\S\ 13).\footnote{Casebook p. 113.}
            Madison was an official, so mandamus was unavailable only if the 
            Judiciary Act was unconstitutional.
            \item Marshall: the Constitution granted the Supreme Court 
            original jurisdiction in only two types of cases (those involving 
            ambassadors, etc., and those in which a state is a party). The 
            Constitution's division of the Court's jurisdiction between 
            original and appellate meant that it did not intend to give 
            Congress the power to expand the Court's original jurisdiction.
            \item Congress therefore did not have the power to expand the 
            Court's original jurisdiction. \S\ 13 of the Judiciary Act was 
            unconstitutional.
            \item \textbf{Constitutional supremacy}: the Supremacy Clause 
            established the Constitution as the ``supreme law of the 
            land.''\footnote{U.S. Const. art. VI, cl. 2.}
            \item \textbf{Judicial review}: limited government requires the 
            Court to review the constitutionality of statutes. ``It is 
            emphatically the province and duty of the judicial department to 
            say what the law is.''\footnote{Casebook p. 116.}
        \end{enumerate}
    \end{enumerate}
\end{enumerate}

\subsubsection{The Marshall Court}

\begin{enumerate}
    \item Marshall established a tradition of outward unanimity (not always 
    followed); tried unsuccessfully to prevent dissents and concurrences---a 
    ``single impersonal opinion.''\footnote{Casebook p. 137}
    % TODO: if this is true, why are per curiam opinions seen as carrying less 
    % weight?
    \item Some argue Marshall furthered Federalist partisan goals by (1) 
    enhancing the federal government's power and (2) designing protections for 
    upper class privileges. Others argue that his moves to increase federal 
    power, increase the judiciary's power, and strengthen property rights 
    against state regulations were political only in the sense of higher 
    political ideology.\footnote{Casebook p. 137.}
\end{enumerate}

\subsection{Theories of Judicial Review}

\subsubsection{Overview}

\begin{enumerate}
    \item \textbf{Judicial interpretation: first or last word?} Is the court's 
    role mainly to legitimize the national government's actions or to 
    invalidate them? Historically, it validates them far more often than 
    not.\footnote{Casebook p. 121 \S\ 1.}
    \item \textbf{Judicial supremacy and judicial finality}: do other branches 
    have to accept Court decisions as authoritative (judicial supremacy)? Does 
    the court get the last word, or do other branches also have the authority 
    to interpret the Constitution (judicial 
    finality/exclusivity)?\footnote{Casebook pp. 121--22 \S\ 2.}
    \item \textbf{Departmentalism}: each branch of government can interpret 
    the constitution.\footnote{Casebook p. 122 \S\ 3.}
    \item \textbf{Constitutional protestantism}: does the ``church'' have the 
    keys to scriptural interpretation, or is scripture accessible to popular 
    understanding?\footnote{Casebook pp. 122-23 \S\ 4 and Sanford Levinson, 
    \emph{Constitutional Faith}.}
    \item \textbf{Popular constitutionalism}: it was widely understood at the 
    time of the founding that popular elections would be the forum for 
    constitutional debate. Courts based their constitutional interpretations 
    as political acts on behalf of the people, rather than a position as 
    privileged interpreters.\footnote{Casebook pp. 123-24 \S\ 5 and Larry 
    Kramer, \emph{The People Themselves}.}
\end{enumerate}

\subsubsection{Precedents}

\begin{enumerate}
    \item The Constitution does not provide for judicial review.
    \item Blackstone: in England, the (unwritten) constitution did not control 
    acts of parliament.\footnote{Casebook p. 124.}
    \item Natural law as the basis---Hamilton, Federalist No. 78: the 
    constitution trumps legislation---``They ought to regulate their decisions 
    by the fundamental laws~.~.~.~''\footnote{Casebook p. 125.}
    \item The framers' intent regarding judicial review is in dispute. It was 
    not firmly established by the Philadelphia convention.
\end{enumerate}

\subsubsection{Judicial Review in a Democratic Polity}

\begin{enumerate}
    \item Frameworks beneath the theories of judicial review:
    \begin{enumerate}
        \item \textbf{Functionalism}: the Court's actions contribute to 
        maintaining a certain kind of American polity.
        \item \textbf{Originalism and textualism}: the Court acts on the framers' 
        intent or on its reading of the text itself.
    \end{enumerate}
\end{enumerate}

\paragraph{The Countermajoritarian Difficulty}

\begin{enumerate}
    \item Alexander Bickel: judicial review ``thwarts the will of the 
    representatives of the actual people of the here and now~.~.~.~'' It's a 
    ``deviant institution in the American democracy.''\footnote{Casebook p. 
    126 and Alexander Bickel, \emph{The Least Dangerous Branch}.}
    \begin{enumerate}
        \item So---why can it exercise broad power?
    \end{enumerate}
    \item Countermajoritarian = elite?
\end{enumerate}

\paragraph{Justification 1: Supervising Inter- and Intra-Governmental 
Relations}

\begin{enumerate}
    \item Judiciary oversees interactions within the federal government and 
    between the federal government and the states.
    \item The Court's power to review state court decisions and the 
    constitutionality of state court decisions is well established. See \S 25 
    of the Federal Judiciary Act of 1789, \emph{Martin v. Hunter's Lessee}, 
    and \emph{Cohens v. Virginia}.\footnote{Casebook p. 128--29.}
\end{enumerate}

\paragraph{Justification 2: Preserving Fundamental Values}

\begin{enumerate}
    \item Bickel: government should protect our immediate needs as well as 
    ``certain enduring values.'' The judiciary is best equipped to enforce 
    these constitutionally protected values because they are insulated from 
    outside pressures---to give a ``sober second thought.''\footnote{Casebook 
    pp. 130--31.}
\end{enumerate}

\paragraph{Protecting the Integrity of Democratic Processes}

\begin{enumerate}
    \item John Hart Ely (``process-defect'' or ``process perfecting'' theory): 
    rather than preserve fundamental values, the Court should base its 
    decisions on promoting democratic participation and representation. The 
    Court should scrutinize legislation that restricts processes of political 
    accountability or that prejudices minorities. In other words, make sure 
    democracy works.  The Court need not worry about values---it should only 
    protect the process.
    \begin{enumerate}
        \item Leaves unchallenged the existing set of values. Can the Court 
        possibly \emph{not} base its decisions on values? Cf. \emph{Roe v. 
        Wade}.
        \item What if society is fundamentally unfair? What if a smooth 
        process ratifies existing unfairness?
    \end{enumerate}
\end{enumerate}

\paragraph{The Countermajoritarian Difficulty Challenged}

\begin{enumerate}
    \item There are other countermajoritarian forces in our governmental 
    structure---e.g., equal representation for all states regardless of 
    population in the Senate (and in the House in cases of Electoral College 
    deadlock), the filibuster, the presidential veto.\footnote{Casebook 
    p. 132.} In addition, the Treaty Clause\footnote{U.S. Const. art. II, \S\ 
    2} requires a two-thirds Senate vote to ratify treaties and Article V 
    requires a two-thirds in both houses to propose an amendment and 
    three-fourths of states to ratify.
    \item Others have challenged the idea that judicial review poses a 
    countermajoritarian difficulty at all.
    \begin{enumerate}
        \item Robert Dahl: in other political contexts, ``\emph{minorities} 
        rule'' by aggregating together. The same happens on the Court. The 
        high turnover rate of Justices makes it unlikely that the Court would 
        ``systematically thwart[] congressional policy.''\footnote{Casebook p. 
        133.} Where the court has struck down significant legislation, it 
        almost always reflected national consensus, or else its decisions were 
        quickly reversed.
        \item Martin Shapiro: a wide range of countermajoritarian forces 
        influence the other branches---``the congressional committee system, 
        the role of seniority, the power of lobbyists, the presidential 
        nominating conventions, the Electoral College, the myriad federal 
        agencies, and the relations among those agencies, their parallel 
        congressional committees, and the industries subject to agency 
        regulation~.~.~.~''\footnote{Casebook pp. 134--35.} The executive and 
        legislative branches are amalgams of majoritarianism and 
        anti-majoritarianism.
        \item Mark Graber: legislatures send political hot potatoes to courts 
        to avoid political difficulties. The Court rarely makes decisions that 
        are opposed to the political majority.\footnote{Casebook p. 135.}
    \end{enumerate}
\end{enumerate}

\paragraph{Lifetime Tenure}

\begin{enumerate}
    \item Derives from ``good Behavior'' in Article III, \S\ 1. One problem is 
    that Justices may cling to outdated thinking, which could be solved with 
    term limits roughly mirroring the current average turnover 
    rate.\footnote{Casebook pp. 135--36.} Another argument is that judges are 
    not, in fact, bound by strict rules and precedent.\footnote{Casebook p. 
    136.}
\end{enumerate}
